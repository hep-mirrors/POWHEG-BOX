\documentclass[paper]{JHEP3}
\usepackage{epsfig}
\usepackage{amsmath}
\usepackage{amssymb}
\usepackage{amsbsy}
\usepackage{bbm}
\usepackage{enumerate}
\usepackage{url}

\bibliographystyle{JHEP}

%\documentclass{article}
%\usepackage{amsmath,amssymb,enumerate,epsfig}

%%%%%%%%%% Start TeXmacs macros
%\newcommand{\tmmathbf}[1]{\ensuremath{\boldsymbol{#1}}}
\newlength{\wfig}
\wfig=0.49\textwidth
\newlength{\hfig}
\hfig=0.38\textwidth
\newlength{\hfigs}
\hfigs=0.36\textwidth

\newcommand{\tmmathbf}[1]{\ensuremath{{\bf #1}}}
\def\tr{\mathop{\rm tr}\nolimits}
%\newcommand{\tmop}[1]{\ensuremath{\operatorname{#1}}}
%\newcommand{\tmtextit}[1]{{\itshape{#1}}}
\newcommand{\tmtexttt}[1]{{\ttfamily{#1}}}
\newcommand{\tmtextrm}[1]{{\rmfamily{#1}}}
%\newenvironment{enumeratenumeric}{\begin{enumerate}[1.] }{\end{enumerate}}
\newcommand{\tmop}[1]{\ensuremath{{\rm #1}}}
\newcommand{\tmtextit}[1]{{\itshape{#1}}}
\newenvironment{enumeratenumeric}{\begin{enumerate}}{\end{enumerate}}
\newenvironment{itemizedot}{\begin{itemize}
\renewcommand{\labelitemi}{$\bullet$}
\renewcommand{\labelitemii}{$\bullet$}
\renewcommand{\labelitemiii}{$\bullet$}
\renewcommand{\labelitemiv}{$\bullet$}}{\end{itemize}} 
%%%%%%%%%% End TeXmacs macros

\def\beq{\begin{equation}}
\def\beqn{\begin{eqnarray}}
\def\eeq{\end{equation}}
\def\eeqn{\end{eqnarray}}
\def\abs#1{\left|#1\right|}
\def\bentarrow{\:\raisebox{1.3ex}{\rlap{$\vert$}}\!\longrightarrow}
\def\FWeq#1{eq.~({\bf I}.#1)}
\def\FNWeq#1{eq.~({\bf II}.#1)}
\def\FKSeq#1{eq.~({\bf FKS}.#1)}
\def\half{\frac{1}{2}}

\newcommand\FKS{Frixione, Kunszt and Signer}
\newcommand\CS{Catani and Seymour}

\newcommand\HERWIG{{\tt HERWIG}}
\newcommand\HWpp{{\tt HERWIG++}}
\newcommand\PYTHIA{{\tt PYTHIA}}
\newcommand\ARIADNE{{\tt ARIADNE}}
\newcommand\SISCONE{{\tt SISCONE}}
\newcommand\FASTKT{{\tt KT}}
\newcommand\ALPGEN{{\tt ALPGEN}}
\newcommand\HqT{{\tt HqT}}

\newcommand\Phirad{\Phi_{\rm rad}}

\newcommand\LambdaQCD{\Lambda_{\scriptscriptstyle QCD}}
\newcommand\mZ{m_{\sss\rm Z}}
\newcommand\muf{\mu_{\sss\rm F}}
\newcommand\mur{\mu_{\sss\rm R}}
\newcommand\mTH{m_{\sss   T}^{{\sss H}}}
\newcommand\mH{m_{\sss  H}}
\newcommand\HThat{\hat{H}_{\sss  T}}
\newcommand\ptrelone{p_{\sss T}^{\sss {\rm rel},j_1}}
\newcommand\ptreltwo{p_{\sss T}^{\sss {\rm rel},j_2}}
\newcommand\ptrel{p_{\sss T}^{\sss {\rm rel},j}}
\newcommand\ptH{p_{\sss T H}}
\newcommand\MINT{{\tt MINT}}
\newcommand\BASES{{\tt BASES}}
\newcommand\SPRING{{\tt SPRING}}

\def\IN{{\small IN}}
\def\OUT{{\small OUT}}
\def\INm{{\sss\rm IN}}
\def\OUTm{{\sss\rm OUT}}

%\def\tk{\widetilde k}
\def\tk{\bar k}
\def\tx{\widetilde x}



\def\lq{\left[} 
\def\rq{\right]} 
\def\rg{\right\}} 
\def\lg{\left\{} 
\def\({\left(} 
\def\){\right)} 
 
\def\REF#1{#1}


\def\KFS#1{K^{{#1}}_{\scriptscriptstyle\rm F\!.S\!.}}
\def\HFS#1{H_{{#1}}^{\scriptscriptstyle\rm F\!.S\!.}}
\def\Kab{\KFS{ab}}
\def\Hba{\HFS{ba}}
\def\Kcb{\KFS{cb}}
\def\Kac{\KFS{ac}}
\def\Hac{\HFS{ac}}
\def\Hcb{\HFS{cb}}
\def\Hab{\HFS{ab}}
\def\cm{{\cal M}}
\def\bom#1{{\mbox{\boldmath $#1$}}}
\def\ket#1{|{#1}>}
\def\bra#1{<{#1}|}

\def\th{\theta}

%% initial-state labels

\newcommand\sss{\mathchoice%
{\displaystyle}%
{\scriptstyle}%
{\scriptscriptstyle}%
{\scriptscriptstyle}%
}

\newcommand\nplus{\oplus}
\newcommand\nminus{\ominus}

\newcommand\splus{{\sss \nplus}}
\newcommand\sminus{{\sss \nminus}}

\newcommand\splusminus{{\mathchoice%
{\vplusminus\displaystyle}%
{\vplusminus\scriptstyle}%
{\vplusminus\scriptscriptstyle}%
{\vplusminus\scriptscriptstyle}%
}}

\newcommand\sminusplus{{\mathchoice%
{\vminusplus\displaystyle}%
{\vminusplus\scriptstyle}%
{\vminusplus\scriptscriptstyle}%
{\vminusplus\scriptscriptstyle}%
}}

\newdimen\hbigcirc
\newdimen\wbigcirc

\newdimen\figwidth

\newcommand\captskip{\vskip -0.7cm}

%\figwidth=14cm 
\figwidth=\textwidth

%\newcommand\vplusminus[1]{%
%\settoheight{\hbigcirc}{$#1\bigcirc$}%
%\settowidth{\wbigcirc}{$#1\bigcirc$}%
%\makebox[\wbigcirc]{\raisebox{0.1\hbigcirc}%
%{\makebox[0pt]{$#1+$}}\raisebox{-0.25\hbigcirc}%
%{\makebox[0pt]{$#1-$}}\makebox[0pt]{$#1\bigcirc$}}%
%}

\newcommand\vplusminus[1]{%
\settoheight{\hbigcirc}{$#1\bigcirc$}%
\settowidth{\wbigcirc}{$#1\bigcirc$}%
\makebox[\wbigcirc]{%
\makebox[0pt]{\rule[0.4\hbigcirc]{0.5\wbigcirc}{0.05\hbigcirc}}%
\makebox[0pt]{\rule[0.1\hbigcirc]{0.5\wbigcirc}{0.05\hbigcirc}}%
\makebox[0pt]{\rule[0.1\hbigcirc]{0.05\wbigcirc}{0.6\hbigcirc}}%
\makebox[0pt]{$#1\bigcirc$}}%
}


\newcommand\vminusplus[1]{%
\settoheight{\hbigcirc}{$#1\bigcirc$}%
\settowidth{\wbigcirc}{$#1\bigcirc$}%
\makebox[\wbigcirc]{%
\makebox[0pt]{\rule[0.2\hbigcirc]{0.5\wbigcirc}{0.05\hbigcirc}}%
\makebox[0pt]{\rule[0.5\hbigcirc]{0.5\wbigcirc}{0.05\hbigcirc}}%
\makebox[0pt]{\rule[-0.1\hbigcirc]{0.05\wbigcirc}{0.6\hbigcirc}}%
\makebox[0pt]{$#1\bigcirc$}}%
}

\newcommand\xplus{x_\splus}
\newcommand\xminus{x_\sminus}
\newcommand\xplusminus{x_\splusminus}
\newcommand\kplus{k_\splus}
\newcommand\kminus{k_\sminus}
\newcommand\Kplus{K_\splus}
\newcommand\Kminus{K_\sminus}
\newcommand\bxplus{\bar{x}_\splus}
\newcommand\bxminus{\bar{x}_\sminus}
\newcommand\bxplusminus{\bar{x}_\splusminus}
%

\newcommand\fmo{{f_{\splus}}}
\newcommand\fmop{{f_{\splus}'}}
\newcommand\fmt{{f_{\sminus}}}

\newcommand\ep{\epsilon}
\newcommand\clH{{\mathbb H}}
\newcommand\clS{{\mathbb S}}
\newcommand\EVprjmap{{\cal P}_{\clH\to\clS}}
\newcommand\as{\alpha_{\sss\rm S}}
\newcommand\asotpi{\frac{\as}{2\pi}}
\newcommand\Lum{{\cal L}}
\newcommand\Lumt{\tilde{\cal L}}
\newcommand\orda{{\cal O}(\as)}
\newcommand\ordaz{{\cal O}(\as^0)}
\newcommand\ordat{{\cal O}(\as^2)}
\newcommand\ordaB{{\cal O}(\as^m)}
\newcommand\ordaBpo{{\cal O}(\as^{m+1})}
\newcommand\pt{p_{\sss T}}
\newcommand\pT{p_{\sss T}}
\newcommand\ptmin{{p_{\sss\rm T\min}}}
\newcommand\pzero{p_{\sss\rm 0}}
\newcommand\kt{k_{\sss\rm T}}
\newcommand\kperp{k_\perp}
\newcommand\tildept{\tilde{p}_{\sss\rm T}}
\newcommand\tildekt{\tilde{k}_{\sss\rm T}}
\newcommand\tildektmaxsq{{\tilde{k}_{\sss\rm T,\,max}^2}}
\newcommand\ximax{{\xi_{\sss\rm{M}}}}
\newcommand\boost{\mathbb B}
\newcommand\xitilde{\tilde{\xi}}
\newcommand\ytilde{\tilde{y}}
%\newcommand\mat{{\cal M}}
%\newcommand\matB{{\cal M}^{(b)}}
%\newcommand\matR{{\cal M}^{(r)}}
%\newcommand\matRS{\widehat{\cal M}^{(r)}}
%\newcommand\matVb{{\cal M}^{(v)}_{\sss B}}
\newcommand\matB{{\cal B}}
\newcommand\matR{{\cal R}}
\newcommand\matRi{{\cal R}^{(\ctindex)}}
\newcommand\matRS{\hat{\matR}}
\newcommand\matVb{{\cal V}_{\bare}}
\newcommand\matMCs{{\cal M}}
\newcommand\matI{{\cal I}}
%\newcommand\matSV{{\cal M}^{(sv)}}
\newcommand\matSV{{\cal V}}

%\newcommand\matCb{{\cal M}^{(c)}_{\sss B}}
%\newcommand\matCpb{{\cal M}^{(c+)}_{\sss B}}
%\newcommand\matCmb{{\cal M}^{(c-)}_{\sss B}}
%\newcommand\matCpmb{{\cal M}^{(c\pm)}_{\sss B}}
%\newcommand\matCpm{{\cal M}^{(c\pm)}}
\newcommand\matC{{\cal G}}
\newcommand\bare{{\rm b}}
\newcommand\matCb{{\cal G}_\bare}
\newcommand\matCpb{{\cal G}_{\splus,\bare}}
\newcommand\matCmb{{\cal G}_{\sminus,\bare}}
\newcommand\matCpmb{{\cal G}_{\splusminus,\bare}}
\newcommand\matCp{{\cal G}_\splus}
\newcommand\matCm{{\cal G}_\sminus }
\newcommand\matCpm{{\cal G}_{\splusminus}}

\newcommand\ctindex{\alpha}
\newcommand\ctindr{{\alpha_{\sss\rm r}}}
\newcommand\ctindpm{{\alpha_{\splusminus}}}
\newcommand\ctindp{{\alpha_{\splus}}}
\newcommand\ctindm{{\alpha_{\sminus}}}
\newcommand\matct{{\cal C}}
\newcommand\matcti{{\cal C}^{(\ctindex)}}
\newcommand\matctiP{\bar{\cal C}^{(\ctindex)}}

\newcommand\GenSh{{\cal F}}
\newcommand\GenShV{{\cal F}_{\sss\rm V}}
\newcommand\GenShVT{{\cal F}_{\sss\rm VT}}
\newcommand\GenShMC{{\cal F}_{\sss\rm MC}}
\newcommand\GenShMCatNLO{{\cal F}_{\sss\rm MC@NLO}}
\newcommand\GenShPOWHEG{{\cal F}_{\sss\rm POWHEG}}

\newcommand\PSnpo{\Phi_{n+1}}
\newcommand\PSn{\Phi_n}
\newcommand\Phit{\tilde{\Phi}}
%\newcommand\MapNLOi{{\bf M}^{(i,-)}}
%\newcommand\MapMCisr{{\bf S}^{\sss (ISR,+)}}
%\newcommand\MapMCfsr{{\bf S}^{\sss (FSR,+)}}
\newcommand\MapNLOi{{\bf M}^{(\ctindex)}}
\newcommand\MapNLOisro{{\bf M}^{\sss\rm (ISR_\splus)}}
\newcommand\MapNLOisrt{{\bf M}^{\sss\rm (ISR_\sminus )}}
\newcommand\MapNLOfsr{{\bf M}^{\sss\rm (FSR)}}
\newcommand\MapMCisr{{\bf S}^{\sss\rm (ISR)}}
\newcommand\MapMCfsr{{\bf S}^{\sss\rm (FSR)}}

%\newcommand\Kinnpo{{\bf K}_{n+1}}
%\newcommand\Kinn{{\bf K}_n}
%\newcommand\Kinnc{{\bf K}_n^c}
%\newcommand\Kinncp{{\bf K}_n^{c+}}
%\newcommand\Kinncm{{\bf K}_n^{c-}}
%\newcommand\Kinncpm{{\bf K}_n^{c\pm}}
%\newcommand\Kinm{{\bf K}_m}
\newcommand\Kinnpo{{\bf \Phi}_{n+1}}
\newcommand\Kinn{{\bf \Phi}_n}
\newcommand\BKinn{{\bf \bar{\Phi}}_n}
\newcommand\BKinm{{\bf \bar{\Phi}}_m}
\newcommand\Kinnc{{\bf \Phi}_n^c}
\newcommand\Kinncp{{\bf \Phi}_{n,\splus}}
\newcommand\Kinncm{{\bf \Phi}_{n,\sminus}}
\newcommand\Kinncpm{{\bf \Phi}_{n,\splusminus}}
\newcommand\BKinncp{\bar{\bf \Phi}_{n,\splus}}
\newcommand\BKinncm{\bar{\bf \Phi}_{n,\sminus}}
\newcommand\BKinncpm{\bar{\bf \Phi}_{n,\splusminus}}
\newcommand\Kinm{{\bf \Phi}_m}
\newcommand\Kinmpo{{\bf \Phi}_{m+1}}
\newcommand\Kinnt{{\bf \tilde{\Phi}}_n}
\newcommand\Kinnpot{{\bf \tilde{\Phi}}_{n+1}}

\newcommand\KinnpoMC{{\bf \Phi}_{n+1}^{({\sss\rm MC})}}
\newcommand\KinnMC{{\bf \Phi}_n^{({\sss\rm MC})}}
\newcommand\KinncpMC{{\bf \Phi}_{n,\splus}^{({\sss\rm MC})}}
\newcommand\KinncmMC{{\bf \Phi}_{n,\sminus}^{({\sss\rm MC})}}

\newcommand\KinnPH{{\bf \Phi}_n^{({\sss\rm PH})}}
\newcommand\KinncpPH{{\bf \Phi}_{n,\splus}^{({\sss\rm PH})}}
\newcommand\KinncmPH{{\bf \Phi}_{n,\sminus}^{({\sss\rm PH})}}
\newcommand\KinntPH{{\bf \tilde{\Phi}}_n^{({\sss\rm PH})}}

\newcommand\stepf{\theta}
\newcommand\Dfun{{\cal D}}
\newcommand\Sfun{{\cal S}}
\newcommand\Sin{{\cal S}^{(\INm)}}
\newcommand\Sout{{\cal S}^{(\OUTm)}}
%\newcommand\Sz{{\cal S}^{(0)}}
%\newcommand\So{{\cal S}^{(1)}}
%\newcommand\Szi{{\cal S}_i^{(0)}}
%\newcommand\Soij{{\cal S}_{ij}^{(1)}}
\newcommand\Sz{{\cal S}}
\newcommand\So{{\cal S}}
\newcommand\Szi{{\cal S}_i}
\newcommand\Szip{{\cal S}_i^\splus}
\newcommand\Szim{{\cal S}_i^\sminus}
\newcommand\Szipm{{\cal S}_i^\splusminus}
\newcommand\Szimp{{\cal S}_i^\sminusplus}
\newcommand\Soij{{\cal S}_{ij}}
\newcommand\Soji{{\cal S}_{ji}}
\newcommand\mydot{\!\cdot\!}
\newcommand\xiic{\left(\frac{1}{\xi_i}\right)_{\!\!\xi_c}}
\newcommand\xic{\left(\frac{1}{\xi}\right)_{\!\!\xi_c}}
\newcommand\lxic{\left(\frac{\log\xi}{\xi}\right)_{\!\!\xi_c}}
\newcommand\omzc{\left(\frac{1}{1-z}\right)_{\!\!\xi_c}}
\newcommand\lomzc{\left(\frac{\log(1-z)}{1-z}\right)_{\!\!\xi_c}}
\newcommand\omyid{\left(\frac{1}{1-y_i}\right)_{\!\!\deltaI}}
\newcommand\opyid{\left(\frac{1}{1+y_i}\right)_{\!\!\deltaI}}
\newcommand\ompyid{\left(\frac{1}{1\mp y_i}\right)_{\!\!\deltaI}}
\newcommand\omyjd{\left(\frac{1}{1-y_j}\right)_{\!\!\deltaO}}
\newcommand\omyijd{\left(\frac{1}{1-y_{ij}}\right)_{\!\!\deltaO}}
\newcommand\ompyd{\left(\frac{1}{1\mp y}\right)_{\!\!\delta}}
\newcommand\Prec{P_{\rm rec}}
\newcommand\Mrec{M_{\rm rec}}
\newcommand\Pboost{K_{\rm rec}}
%\newcommand\boosted[1]{{\underline #1}}
%\newcommand\mmod[1]{\boldsymbol{#1}}
\newcommand\mmod[1]{\underline{#1}}
\newcommand\MCatNLO{{\tt MC@NLO}}
\newcommand\pMCatNLO{{pMC@NLO}}
\newcommand\xicut{\xi_{c}}
\newcommand\deltaI{\delta_{\sss\rm I}}
\newcommand\deltaO{\delta_{\sss\rm O}}
\newcommand\MSB{{\rm \overline{MS}}}
\newcommand\DIG{{\rm DIS}_\gamma}
\newcommand\CA{C_{\sss\rm A}}
\newcommand\DA{D_{\sss\rm A}}
\newcommand\CF{C_{\sss\rm F}}
\newcommand\TF{T_{\sss\rm F}}
\newcommand\NC{N_{\rm c}}
\newcommand\NCt{N_{\rm c}^2}
\newcommand\NF{n_{\rm f}}

%\newcommand\APl{{<}}
\newcommand\APl{{reg}}
\newcommand\APreg{\hat{P}}
\newcommand\indsing{{i_s}}
\newcommand\fr{{f_r}}
\newcommand\fb{{f_b}}
\newcommand\Fb{{F_b}}
\newcommand\POWHEG{{\tt POWHEG}}
\newcommand\POWHEGBOX{{\tt POWHEG BOX}}
\newcommand\MG{{\tt MadGraph4}}
\newcommand\HELAS{{\tt HELAS}}
\newcommand\MCFM{{\tt MCFM}}
\newcommand\MadDipole{{\tt MadDipole}}
\newcommand\MadFKS{{\tt MadFKS}}
\newcommand\MGS{{\tt MadGraphStuff}}
\newcommand\ttilde{\raise.17ex\hbox{$\scriptstyle\mathtt{\sim}$}}

\newcommand\Rad{\Phi_{\rm rad}}
\newcommand\Xrad{X_{\rm rad}}
%\newcommand\BRad{\bar{\Phi}_{\rm rad}}


\newcommand\frindsing{{\ctindr}}

\newcommand\qb{\bar{q}}
\newcommand\RED{}
\def\ord#1{{\cal O}\(#1\)}
\def\ga{\gamma}
\def\sirave{\overline{\sigma}_{\rm real} }
\def\siNLOave{\overline{\sigma}_{\rm NLO} }
\def\siMSBave{\overline{\sigma}_{\rm NLO}^{\rm \overline{MS}}}
\def\sicoll{\sigma^{\rm coll}_{\rm ct}}
\def\sibave{\overline{\sigma}_{\rm Born} }
\def\sivave{\overline{\sigma}_{\rm virt} }

%\newcommand\xpm{{x_{\splus\sminus}}}
\newcommand\xpm{x}
\newcommand\xipm{{x_{i,\splus\sminus}}}
\newcommand\vplus{{\tilde{v}_\splus}}
\newcommand\vminus{{\tilde{v}_\sminus}}
\newcommand\vpm{{\tilde{v}_\splusminus}}

\newcount\minutes 
\newcount\scratch 
 
\def\timestamp{% 
\scratch=\time 
\divide\scratch by 60 
\edef\hours{\the\scratch} 
\multiply\scratch by 60 
\minutes=\time 
\advance\minutes by -\scratch 
%\the \month/\the\day$\,$ 
---$\,$\hours:\null 
\ifnum\minutes< 10 0\fi 
\the\minutes} 
 

%%%%%%%%%%%%%%%%%%%%%%%%%%%%%%%%%%%%%%%%%%%%%%%%%%%%%%%%%%%%%%%%%%%%%%%%%%
\preprint{FERMILAB-PUB-12-040-T\\
}
\title{NLO Higgs boson production plus one and two jets
% via gluon fusion matched with shower 
using the POWHEG BOX, MadGraph4 and MCFM}

\author{John Campbell\\
Fermilab, Batavia, IL 60510, USA\\
  E-mail: \email{johnmc@fnal.gov}}

\author{R. Keith Ellis\\
Fermilab, Batavia, IL 60510, USA\\
  E-mail: \email{ellis@fnal.gov}}


\author{Rikkert Frederix\\
  Institut f\"ur Theoretische Physik,
  Universit\"at Z\"urich, Winterthurerstrasse 190, CH-8057 Z\"urich,
  Switzerland\\
  E-mail: \email{frederix@physik.uzh.ch}}


\author{Paolo Nason\\
  INFN, Sezione di Milano-Bicocca,
  Piazza della Scienza 3, 20126 Milan, Italy\\
  E-mail: \email{Paolo.Nason@mib.infn.it}}


\author{Carlo Oleari\\
  Universit\`a di Milano-Bicocca and INFN, Sezione di Milano-Bicocca\\
  Piazza della Scienza 3, 20126 Milan, Italy\\
  E-mail: \email{Carlo.Oleari@mib.infn.it}}


\author{Ciaran Williams\\
Fermilab, Batavia, IL 60510, USA\\
  E-mail: \email{ciaran@fnal.gov}}


\abstract{
  We present a next-to-leading order calculation of Higgs boson production
  plus one and two jets via gluon fusion interfaced to shower Monte Carlo
  programs, implemented according to the \POWHEG{} method.  For this
  implementation we have used a new interface of the \POWHEGBOX{} with \MG{},
  that generates automatically the codes for the Born and real processes.
  The virtual corrections have been taken from the \MCFM{} code.
%  Beside being of great interest by themseves, these two processes have
%  considerable complexity, and thus validate the viability of these new
% interfaces.
  We carry out a simple phenomenological study of our generators, comparing
  them among each other and with fixed next-to-leading order results.
  
%
}
\keywords{QCD, Monte Carlo, NLO Computations, Resummation, Collider Physics
\vfill 
\today \timestamp \hfill 
\vfill 
}

% Enlarge spacing between rows of a table (otherwise the overlines
% are difficult to read
\renewcommand\arraystretch{1.1}

\begin{document}

\section{Introduction}
The search for the Standard Model Higgs boson is entering the endgame phase.
Since the start of the LHC running, the allowed mass range for the standard
model Higgs boson has been greatly curtailed. Furthermore there are
tantalizing, but inconclusive, hints in the remaining low mass
region~\cite{Collaboration:2012si, Chatrchyan:2012tx}. Should these hints be
confirmed by further data, it will be a matter of some urgency to examine the
properties of the new state (or states). In the low mass region, the
Standard Model Higgs boson is predicted to have about ten branching ratios
with branching fractions greater than one per mille, so there will a number
of channels to be studied.

In order to extract the Higgs couplings from these different channels, detailed
information about the production rates will be
required~\cite{Zeppenfeld:2000td, Duhrssen:2004cv, Plehn:2001nj}.
In this context the Higgs + 2 jet process enters principally as an irreducible 
background for the vector boson fusion process. In fact applying cuts 
appropriate for the search for the vector boson fusion process, it is found 
that the Higgs + 2 jet process from gluon fusion contributes more than 
25\% of the signal~\cite{Campbell:2010cz}. 
It is therefore of importance to provide NLO 
predictions for the gluon-fusion initiated Higgs + 2 jet process in a 
form that the experimenters can easily use.

The Higgs boson search channels are often subdivided in this manner, according to
the number of associated jets, for a number of reasons.
First, a differing number of associated jets can imply a
different source of background.  For example, in the search for a Higgs boson
decaying to $W^+ W^-$ the backgrounds have different origins in the
differing jet bins, so it proves advantageous to analyze the different jet
multiplicities separately. Second, in the two jet bin, a new production
channel, vector boson fusion will come into play. For these reasons it is
pressing to provide predictions for these processes in the \POWHEG{}
formalism in such a way that the experimenters can incorporate the best
theoretical information into their analyses.

Phenomenological studies for the production of a Higgs boson in association
with two jets, at the parton level and without a hadron shower, have been
presented for the LHC operating at $\sqrt{s} = 10$~TeV~\cite{Campbell:2010cz}
and $\sqrt{s} = 14$~TeV~\cite{Campbell:2006xx}.

In this work we have implemented Higgs boson plus one and two jet production
in gluon fusion using the \POWHEG{} method. While for the first process a
matched NLO+shower calculation has already appeared in a
publication~\cite{Hoeche:2011fd}, the second one has never been performed before.
We have built our generators using the \POWHEGBOX{}
framework~\cite{Alioli:2010xd}, with the virtual corrections taken from the
\MCFM{} program~\cite{MCFM}, and the Born , colour-correlated Born,
spin-correlated Born and real contributions computed using a new \POWHEGBOX{}
interface to the \MG{}~\cite{Stelzer:1994ta, Alwall:2007st} program, which is
fully generic and can be applied to any process. Thus, the aim of this paper
is twofold:
\begin{itemize}
\item To present a new generic interface of \MG{} to the \POWHEGBOX{} that
  allows for the automatic generation of code for the Born, Born colour- and
  spin-correlated squared amplitudes, for the real-radiation amplitude and for
  the Born-colour flow at leading colour approximation. These  ingredients
  are all needed in the implementation of a new process in the \POWHEGBOX{}. This
  interface can be used for any process that can be generated with \MG{}.
  Thanks to this interface, in order to construct a \POWHEGBOX{} generator,
  one needs only to provide the Born phase space, and the virtual
  corrections.
\item  To illustrate the Higgs plus one and two jets generators, by comparing
  their output to the corresponding NLO results and amongst themselves.
\end{itemize}





\section{The interface to \MG}
\label{sec:MGinterface}
We have built a generic interface of the \POWHEGBOX{} to the \MG{} package,
for the construction of all the ingredients listed in the introduction of
ref.~\cite{Alioli:2010xd}, with the exception of the Born phase space and of
the virtual matrix elements.

The \MG\ package~\cite{Stelzer:1994ta, Alwall:2007st} generates squared
tree-level matrix elements in an automatic way. It is based on Feynman
diagrams which are constructed \HELAS{} subroutines~\cite{Murayama:1992gi}
using as building blocks and that are combined into colour-ordered amplitudes
which are written in a FORTRAN code. Furthermore, the
\MadDipole~\cite{Frederix:2008hu,Frederix:2010cj,Gehrmann:2010ry} extension
of \MG\ and the \MadFKS\ tool~\cite{Frederix:2009yq} can generate the Born
spin- and color-correlated squared amplitudes, which makes the \MG\ framework
an ideal environment to complement the \POWHEGBOX\ in the creation of all
tree-level squared matrix elements. In this section we describe a newly
developed interface between these two codes.


For a given process, the routines and parameters that are generated by
\MG\ for the \POWHEGBOX\ are (see ref.~\cite{Alioli:2010xd} for more
details on the notation and conventions used)
\begin{itemize} 
\item the multiplicity of the Born and real emission processes,
  \verb|nlegborn| and \verb|nlegreal|, respectively, defined in the
  \verb|nlegborn.h| include file;

\item the list of Born and real-emission flavour structures\\
  \verb|flst_born(k=1,nlegborn,j=1:flst_nborn)|,   \\
  \verb|flst_real(k=1,nlegreal,j=1:flst_nreal)|,   \\
  respectively, to be initialised by a call to the \verb|init_processes|
  routine;

\item the routine\\
  \verb|setborn(p(0:3,1:nlegborn),bflav(1:nlegborn),born,|\\
  \verb|        bornjk(1:nlegborn,1:nlegborn),bmunu(0:3,0:3,1:nlegborn))|\\
   which computes, for a given set of four-momenta \verb|p| and flavour
   structure \verb|bflav|, the Born squared matrix element \verb|born|, the
   colour-correlated, \verb|bornjk|, and the spin-correlated one,
   \verb|bmunu|;

\item the routine that computes the real emission squared matrix elements
  \verb|amp2| (stripped of a factor $\as/(2\pi)$), for a given momentum
  configuration \verb|p| and flavour structure \verb|rflav|\\
  \verb|setreal(p(0:3,nlegreal),rflav(1:nlegreal),amp2)|;

\item a colour flow assignment to the Born squared matrix elements, in the
  leading colour approximation, needed by the showering programs. The colour
  assignment is done on statistical grounds, based on information that is
  cached during the call to the squared matrix elements, by the routine
  \verb|borncolour_lh|. In addition to the this routine, the interface
  provides also a similar one for the real squared matrix elements,
  \verb|realcolour_lh|, not used by the present version of the \POWHEGBOX{},
  that uses its own method to assign the colour to the real-radiation matrix
  elements, as discussed in sec.~8 of ref.~\cite{Alioli:2010xd}.

\item an interface to the Les Houches parameter input card to specify the
  physics model parameters through a routine \verb|init_couplings|.
\end{itemize}
%
In distinction to the default way of generating amplitudes with \MG,
all the squared matrix elements for the various flavour structures are
written in the same directory, making sure that the files and routines
have different names. This allows one to compile the code into a
single library that contains all the matrix elements for all the
flavour structures. An interface is provided that concatenates the
flavour vectors \verb|bflav| and \verb|rflav| into strings, which are
used as unique identifiers of each of the squared matrix elements.


More technical details on the use of the \MG{} interface to the \POWHEGBOX{}
can be found in Appendix~\ref{app:tech_details}.


\section{Higgs boson production in gluon fusion}
In this paper we present results for the production of a Higgs boson in
association with one and two jets, at NLO, interfaced with a parton shower
using the \POWHEG{} method.  The basic hard cross sections for a Higgs boson
in association with three, four or five partons, needed for the Higgs boson
plus one and two jet cross sections at NLO are calculated using an effective
Lagrangian to express the coupling of the gluons to the Higgs
field~\cite{Wilczek:1977zn}
\begin{equation} 
\label{EffLag}
\mathcal{L}_{\sss H} = \frac{C}{2} \, H\,\tr
G_{\mu\nu}\,G^{\mu\nu}\,,
\end{equation}
where the trace is over the color degrees of freedom.  At the order required
in this paper, the coefficient $C$ is given in the $\overline{\rm MS}$ scheme
by~\cite{Djouadi:1991tka,Dawson:1990zj}
\begin{equation}
C =\frac{\as}{6 \pi v} \( 1 +\frac{11}{4 \pi} \as\)
 + {\cal O}(\as^3) \,,
\end{equation}
where $v$ is the vacuum expectation value of the Higgs field, $v = 246$ GeV.

%As explained in the following sections, the tree-level matrix elements are
%generated automatically using a special purpose version of \MG{}.  The
%one-loop amplitudes for the Higgs boson plus three and four parton processes
%are extracted from \MCFM{} as described below.

We have used the automatic interface described in sec.~\ref{sec:MGinterface} to
generate the code for the Born, Born colour- and spin-correlated amplitudes
and for the real cross section. In appendix~\ref{app:tech_details} we give
some more detailes on the adopted procedure. The
one-loop amplitudes for the Higgs boson plus three and four parton processes
are extracted from \MCFM{} as described below.

\subsection{Virtual cross section}
The complete set of one-loop amplitudes for all Higgs + 4 parton processes
are now available~\cite{Berger:2006sh, Badger:2006us, Badger:2007si,Glover:2008ffa, Badger:2009hw, Dixon:2009uk, Badger:2009vh}. 
These formulae have been implemented into \MCFM{} ~\cite{Campbell:2010cz}. 
  
The interface between \MCFM{} and the \POWHEGBOX{} is fairly straightforward.
The virtual pieces share the same phase space as the Born
contributions. Since in the \POWHEG{} implementation the Born, real and
subtraction terms are provided elsewhere, \MCFM{} needs only to return the
pure virtual contributions.  The interface to \MCFM{} transfers the
electroweak parameters, scale and scheme choices.  Once these are
established, calls to the virtual routines of \MCFM{} can be made on an
event-by-event basis.
  
Since the routines that fill the values of the electroweak parameters (and
other information) are generic, this interface could be fairly easily
extended to include other processes currently implemented in \MCFM. However
the normal \MCFM{} routines are designed to return matrix elements that are
summed over the flavour of the final state partons. Therefore in order to
correctly interface to the \POWHEGBOX{} one must make small modifications to
the \MCFM{} code such that it returns the matrix elements for individual
final state flavour combinations.


\section{Implementation of the $Hj$ and $Hjj$ generators}
Using the automated \MG interface described in the previous section,
we have built the routines for the Born, color- and spin-correlated Born,
and the real squared amplitudes, directly in the format required by the
\POWHEGBOX{}. The generation of the amplitudes is fast, taking only a few
minutes for the $Hjj$ process. The virtual corrections were extracted from
the \MCFM{} code. At this point, the only missing ingredient for completing
our generators is the Born phase space.

\subsection{Phase space for the $Hj$ generator}
The phase space routine for the $Hj$ Born process is trivial, since it is
just a two-body phase space. It has been implemented with the possibility
to activate an optional cut on the transverse momentum of the Higgs, and
with an effective importance sampling of the small transverse momentum region.
The cut is necessary if one wants to generated unweighted events. Alternatively
the cut can be set to zero, and the cross section is weighted with a suppression
factor,  that is a function of the underlying Born kinematics. This factor is equal to
\begin{equation}
\label{eq:F_funct}
F=\frac{\pt^2}{\pzero^2+\pt^2}\,,
\end{equation}
where $\pt$ is the Higgs transverse momentum in the underlying Born
configuration and $\pzero$ is set by the user~\footnote{The value of
  $\pzero$, set to 20~GeV for the simulation done in this paper, or the form
  of the suppression factor $F$ in eq.~(\ref{eq:F_funct}), can be changed by
  the user by modifying the {\tt born\_suppression} routine in the {\tt
    Born\_phsp.f} file.}. Events are then generated uniformly in the cross
section times $F$, and with a weight proportional to $1/F$. The normalization
of the weight is such that the cross section for events passing a set of cuts
is given by the sum of all weights for the events passing the cut divided by
the total number of generated events.


\subsection{Phase space for the $Hjj$ generator}
The phase space for the $Hjj$ generator was built using the
same factorized phase space that \POWHEG{} uses for the real kinematics.
In other words, we treat the $Hjj$
Born phase space as the real phase space for the $Hj$ process with one extra
emission.
Furthermore, we have allowed for the possibility to perform importance
sampling in the region where the Born cross section becomes singular.
More in detail, labeling with 1 and 2 the two final state light
partons in the $Hjj$ process, we write the $Hjj$ phase space using the
following identity
\begin{equation}
 d\Phi_{H12}=d\Phi_{H12}\, \frac{N}{d_{12}}\, \frac{E_1}{E_1+E_2}
+d\Phi_{H12} \, \frac{N}{d_{12}}\,\frac{E_2}{E_1+E_2}
+d\Phi_{H12} \, \frac{N}{d_{1}}
+d\Phi_{H12} \, \frac{N}{d_{2}}\,, 
\label{eq:phspspit}
\end{equation}
where $E_i$ is the energy of the $i$-th parton in the center-of-mass frame, and
\begin{equation}
N=\left(\frac{1}{d_{12}}+\frac{1}{d_{1}}+\frac{1}{d_{2}}\right)^{-1}
\end{equation}
where $d_1$, $d_2$ and $d_{12}$ are phase-space functions that vanish respectively
when parton 1 is collinear to the initial state, parton 2 is collinear to
the initial state, and parton 1-2 are collinear to each other. Their precise
form is given in eqs.~(4.23)-(4.26) of ref.~\cite{Alioli:2010xd}.
We then factorize each phase space factor in~(\ref{eq:phspspit}) according
to the formula
\begin{eqnarray}
 d\Phi_{H12}&=&d\Phi_{H2}\,d\Phirad^{(21)} \,\frac{N}{d_{12}}\,\frac{E_1}{E_1+E_2}
+d\Phi_{H1}\, d\Phirad^{(12)} \, \frac{N}{d_{12}}\, \frac{E_2}{E_1+E_2}
\nonumber \\
&&{}+d\Phi_{H2}\, d\Phirad^{(01)} \, \frac{N}{d_{1}}
+d\Phi_{H1}\, d\Phirad^{(02)}\, \frac{N}{d_{2}}\,.
\label{eq:phspspit1}
\end{eqnarray}
The subscripts $H1$ or $H2$ characterize the underlying Born,
while the superscript in $\Phirad$ specifies the radiation process.
Thus, for example, $d\Phi_{H2}$ is the underlying Born phase space of
the Higgs together with parton 2, and $d\Phirad^{(21)}$ is the radiation
phase space corresponding to parton 2 emitting parton 1. The notation
$d\Phirad^{(01)}$ or $d\Phirad^{(02)}$ means that partons 1 or 2 are emitted
by the initial state partons. The factorization of the phase space into
underlying Born and radiation phase space for both initial and final
state radiation is the default one used in
\POWHEG{}, and is described in detail in section~5.1 and~5.2 of
ref.~\cite{Frixione:2007vw}. The decomposition in eq.~(\ref{eq:phspspit1})
is such that appropriate importance sampling is performed in all singular
regions. In fact, the factors in each term damp all but one singular region,
and the corresponding factorized phase space performs importance sampling
precisely in that region. Ideally, the phase space integration should be performed 
with an integrator that can sum over a discrete variable. The \POWHEGBOX{}
integrator~\cite{Nason:2007vt} does not have this feature at the moment.
Thus, we divided the range of one extra integration variable into four segments,
mapping each segment to one of the phase space components.

The phase space of eq.~(\ref{eq:phspspit1}) is not the default one
used in the $Hjj$ generator. It is activated by setting the variable
{\tt fullphsp} in the {\tt powheg\_input} file. The default phase
space is simply
\begin{equation}
d\Phi_{H12}=d\Phi_{H1} \, d\Phirad^{(02)}\,,
\end{equation}
with no importance sampling at all.  We have in fact observed that the loss
of efficiency due to the increased number of calls to the matrix element
routines overwhelms any benefit arising from the improved importance
sampling.

As for the case of the $Hj$ generator, the $Hjj$ generator also includes
the implementation of a Born suppression factor for the suppression of the
singularities of the underlying Born in the
{\tt born\_suppression} routine. It has the form
%\begin{equation}
%F=\frac{\frac{1}{q_0^2}}{\frac{1}{q_1^2}+\frac{1}{q_2^2}+
%\frac{1}{q_{12}^2}+\frac{1}{q_0^2}}\,,
%\end{equation}
\begin{equation}
F=\( \frac{1/\pzero^2}{1/p_{\sss T1}^2+1/p_{\sss T2}^2+1/p_{\sss T12}^2+ 1/\pzero^2}\)^2,
\end{equation}
where $\pzero$ is a parameter that characterizes the minimum jet energy where
some accuracy is required, $p_{\sss T1}$ and $p_{\sss T2}$ are the transverse momenta
(with respect to the beam axis) of the two final state partons, and
\begin{equation}
p_{\sss T12}^2=2 (1-\cos\theta_{12}) \,\frac{E_1^2\,E_2^2}{E_1^2+E_2^2}\,,
\end{equation}
that can be interpreted as the transverse momentum of parton 1 with respect
to parton 2 or vice-versa, depending upon which is the softest.  This
suppression factor can also play the role of a generation cut, if $\pzero$ is
chosen small enough.

\subsection{The \POWHEGBOX{} parameter setting}
We have turned on the {\tt bornzerodamp} flag by default in both the $Hj$ and
the $Hjj$ generators. The purpose of this flag is explained in
ref.~\cite{Alioli:2010xd}.  This results in a considerable speedup of the
$Hjj$ code.  However, no appreciable differences were observed in the
distribution obtained without the {\tt bornzerodamp} option.

The separation of the singular regions is controlled in the \POWHEGBOX{} by
the parameters {\tt par\_diexp} and {\tt par\_dijexp} that are set to $1$ by
default~(see ref.~\cite{Alioli:2010xd}, section~4.7).  In the generators at
hand we have preferred to set them to $2$, which leads to slightly better
stability for the set of distributions that we have considered. We notice
that also in the \POWHEGBOX{} dijet generator~\cite{Alioli:2010xa} these
parameters were set to $2$. Higher values of these parameters correspond to
sharper separation of singular regions.

\section{Phenomenology}
\label{sec:phenomenology}
In this section, we present a phenomenological study of our generators. This
study does not aim to produce results with realistic experimental cuts, but
rather to compare the output of the generators in the description of some key
observables. In particular, we compare the generator results to the fixed NLO
result and with the \POWHEG{} output at the level of the generation of the
hardest radiation (i.e.~before interfacing the result to a
parton shower program), after the shower with no hadronization effects,
and after hadronization.  In the following, we use the notation $H$, $Hj$ and
$Hjj$ to refer to Higgs boson generators that, at the Born level, describe
the production of a Higgs boson plus zero, one and two partons.

We present results for the LHC running at 7~TeV, computed using the CTE6M
parton distribution function~(pdf) set~\cite{Pumplin:2002vw}. 
The same calculations can
easily be performed with any other available set~\cite{Martin:2009iq,Ball:2008by}, but a
study of pdf effects is beyond the scope of the present paper.  Jets are
reconstructed using the anti-$\kt$ jet algorithm~\cite{Cacciari:2008gp}, 
with $R=0.5$ and
default recombination scheme. Cuts of 20, 50 and 100~GeV on the final-state
jets are considered.

The factorization- and renormalization-scale choice deserves some more
detailed discussion.  In view of the large NLO corrections for the processes
at hand, the scale dependence is quite large. This is also a consequence
of the fact that  Higgs boson production in gluon fusion starts at order $\as^2$,
so that the $Hj$ and the $Hjj$ processes are of order $\as^3$ and $\as^4$,
respectively. While for $H$ production the natural 
scale choice is of the order of
the Higgs boson mass, in the case of $Hj$ production one may have a two-scale
problem if the transverse momentum of the jet is much smaller or much larger
than the Higgs boson mass. In addition, besides the Higgs boson mass, one may
also consider the Higgs boson transverse momentum, which may be more
appropriate for very small or very large Higgs boson transverse momentum
$\pt^{\sss H}$, or the Higgs boson transverse mass $\mTH=\sqrt{\mH^2+\pt^2}$, which
may be more appropriate for large Higgs boson transverse momentum. For the
$Hjj$ case, there are even more possibilities. Although a full study of scale
dependence would be very valuable, we will not perform it in the present
work.
In the $H$ case we will  limit ourselves to the fixed Higgs
boson mass scale choice, and $\pt^{\sss H}$ choice at the underlying-Born level.
In the $Hjj$ case we will consider the fixed Higgs
boson mass scale choice, and  the $\HThat$ scale choice
at the underlying Born level, where
\begin{equation}
\HThat=\mTH+\sum_i {\pt}_i
\end{equation}
and ${\pt}_i$ are the final-state parton transverse momenta.
All results shown in the following have been computed with $\mH=120$~GeV.

In the following, we will label ``LHE'' the results obtained at the level of the
\POWHEG{} first emission, ``PY'' the results obtained with the
\POWHEG{}+\PYTHIA{} combination with the hadronization and underlying event
switched off, and with ``NLO'' the pure NLO results.

For some observables, there is overlap among the various generators. For
example, the Higgs boson transverse momentum, as well as the one-jet
multiplicity and the transverse momentum of the leading jet, are described by
both the $H$ and the $Hj$ generators.  The $H$ generator gives a plausible
description for these quantities also at small transverse momenta, since it
provides a resummation of transverse momentum logarithms that the $Hj$
program does not provide.  On the other hand, at large transverse momenta,
the $Hj$ generator has NLO accuracy, something that the $H$ generator does
not have.  For the same quantities, the $Hjj$ generator cannot be used, since
it requires the presence of a second jet. The two-jet multiplicity, as well as
the second hardest jet transverse-momentum distribution, are provided by
both the $Hj$ and the $Hjj$ programs, again with a different level of
accuracy.  A comparison of the generators in the regions where they overlap
will also be carried out.

\subsection{Results for $Hj$ production}
We have simulated $Hj$ production with two scale choices: $\muf=\mur=\mH$ and
the $\pt$ of the underlying Born parton $\muf=\mur=\pt^{\sss\rm UB}$.
All the following plots will come in pairs, with the left plot referring to
$\muf=\mur=\mH$, and the right one to $\muf=\mur=\pt^{\sss\rm UB}$.
We
compare the fixed NLO results, the \POWHEG{} hardest-emission results~(LHE) and
\POWHEG{}+\PYTHIA{}~(PY) ones, where hadronization and underlying event in
\PYTHIA{} have been turned off (see Appendix~\ref{app:PY_setup}).  For a few
physical observables, we show results for three different cuts applied to the
final-state jets: $\pt$ cuts of $20$, $50$ and $100$~GeV.  

\begin{figure}[htb]
\begin{center}
\includegraphics[height=\hfig]{plots/HJ_mult_fixsc} 
\includegraphics[height=\hfig]{plots/HJ_mult_runsc} 
\caption{One- and two-jet multiplicities for the \POWHEG{}
  hardest-emission~(LHE) results, the \POWHEG{} events showered with
  \PYTHIA{}~(PY) and the fixed NLO result~(NLO).  The left plot is computed
  using $\muf=\mur=\mH$, while in the right plot $\muf=\mur=\pt^{\sss\rm
    UB}$.}
\label{fig:HJ_mult}
\end{center}
\end{figure}
In fig.~\ref{fig:HJ_mult} we compare the jet multiplicity for one and
two jets, at NLO, LHE and PY level, for jet cuts of 20, 50 and 100~GeV. We
notice that the one-jet multiplicity is similar in the three results. Understandably 
the \PYTHIA{} multiplicity is smaller than the LHE, due to showering
off the first jet.  More marked differences can be seen in the cross
section of the second jet: here the LHE result is definitely larger than the
NLO one, showing however a very different pattern as a function of the jet
cut, depending upon the scale choice. We will clarify the origin of these
patterns later, when we discuss the transverse-momentum spectrum of the jets.
We point out, however, that the NLO result for the two-jet multiplicity
is of order $\as^4$, and has thus a marked scale dependence. For
$\pT^{\sss j}>20$~GeV, the factorization scale is of the order of $20\;$GeV in the
right plot, while it is the Higgs boson mass in the left plot, which justifies the
large difference in the corresponding NLO results.


\begin{figure}[htb]
\begin{center}
\includegraphics[height=\hfigs]{plots/HJ_H-pt_fixsc}
\includegraphics[height=\hfigs]{plots/HJ_H-pt_runsc} 
\caption{Transverse momentum of the Higgs boson.  The notation is as in
  fig.~\ref{fig:HJ_mult}.  }
\label{fig:HJ_h-pt}
\end{center}
\end{figure}
In fig.~\ref{fig:HJ_h-pt} we show the transverse momentum of the Higgs boson.
The LHE, the PY and the NLO results are in remarkable
agreement for this quantity, which is expected, due its inclusiveness.  
\begin{figure}[htb]
\begin{center}
\includegraphics[height=\hfigs]{plots/HJ_H-y_fixsc}
\includegraphics[height=\hfigs]{plots/HJ_H-y_runsc} 
\caption{Rapidity distribution of the Higgs boson in the $Hj$ process. The
  three sets of curves refer to the three cuts on the hardest jet,
  i.e.~$\pt^{\sss j_1} > 20$, 50 and 100~GeV, from top to bottom respectively. }
\label{fig:HJ_H-y}
\end{center}
\end{figure}
\begin{figure}[htb]
\begin{center}
\includegraphics[height=\hfigs]{plots/HJ_j1-pt_fixsc} \nolinebreak
\includegraphics[height=\hfigs]{plots/HJ_j1-pt_runsc} 
\caption{The hardest jet distribution in the $Hj$ process.}
\label{fig:HJ_j1-pt}
\end{center}
\end{figure}
The Higgs boson rapidity distribution is plotted in fig.~\ref{fig:HJ_H-y}
while the hardest jet $\pt$ in fig.~\ref{fig:HJ_j1-pt}. Both these quantities
display a behaviour similar to the Higgs boson transverse momentum, again
expected due to their inclusiveness.

We now turn to more exclusive quantities, beginning with the transverse
momentum of the second jet, shown in fig.~\ref{fig:HJ_j2-pt}.
\begin{figure}[htb]
\begin{center}
\includegraphics[height=\hfigs]{plots/HJ_j2-pt_fixsc}  \nolinebreak
\includegraphics[height=\hfigs]{plots/HJ_j2-pt_runsc} 
\caption{Transverse-momentum distribution of the second hardest jet in the
  $Hj$ process, with a 20~GeV cut on the transverse momentum of the first
  jet.}
\label{fig:HJ_j2-pt}
\end{center}
\end{figure}
The characteristic NLO behaviour, diverging at small transverse momenta, is
clearly visible in the insert. The LHE result displays, instead, the typical
Sudakov damping at small $\pT$, compensated by an increase at larger
$\pT$. The fully showered result is smaller, since further showering degrades
the second jet transverse momentum. The two-jet multiplicity displayed in
fig.~\ref{fig:HJ_mult} is consistent with the value of the cross section
around $\pT=20\;$GeV in fig.~\ref{fig:HJ_h-pt}.  From the upper insert in the
figure, we see that, at this value of transverse momentum, the LHE cross
section is above the NLO one for the fixed-scale choice (left plot), while it
is near to it for the running-scale choice (right plot). For larger $\pT$
cuts, the LHE cross section remains above the NLO one in both cases, with a
growing trend in the running scale case.  The comparison of the fixed-scale
and running-scale choice in the figure also has some subtle features that
should be remarked upon. The running scale choice yields smaller/larger
scales for small/large second jet $\pT$. The LO cross section is larger for
smaller scale choices, so that the NLO corrections are smaller, and
vice-versa for larger scale choices. Thus, for small $\pT$, the NLO
corrections are smaller than for large $\pT$.  In \POWHEG{}, one first
generates the underlying Born configuration, according to the cross section
$\bar{B}$, which is the total inclusive cross section at fixed underlying
Born configuration. The radiation kinematics is then generated using a shower
technique. It thus turns out that the whole distribution for the radiation is
amplified by a $\bar{B}/B$ $K$-factor~\cite{Alioli:2008tz, Nason:2010ap,
  Nason:2012pr}.  In this case, due to the growth of the NLO corrections as a
function of $\pT$, the amplification of the radiated-jet distribution due to
the $\bar{B}/B$ $K$-factor increases as a function of the transverse
momentum, a trend that is visible in the right figure. Conversely, no such
effect is present for fixed scales. However, in this last case, one should
recall that, in the LHE events, one power of $\as$ is effectively evaluated
at the transverse momentum of the jet rather than at the fixed scale, thus
yielding a decrease in the cross section which is also visible in the left
plot.
%
The larger value of the LHE cross section with respect to the NLO one is
related to the large $K$-factor, i.e.~is due to the fact that the hardest
radiation is amplified by a factor $\bar{B}/B$.

\begin{figure}[htb]
\begin{center}
\includegraphics[height=\hfigs]{plots/HJ_j2-pt_fixsc_bornonly} \nolinebreak
\includegraphics[height=\hfigs]{plots/HJ_j2-pt_runsc_bornonly} 
\caption{Transverse momentum distribution of the second hardest jet using the
  $\bar{B}\to B$ option.}
\label{fig:HJ_j2-pt_bornonly}
\end{center}
\end{figure}
Figure~\ref{fig:HJ_j2-pt_bornonly} is similar to fig.~\ref{fig:HJ_j2-pt}
except that the LHE result is obtained setting the {\tt bornonly} flag to
true in the \POWHEGBOX{} input file. When this flag is true, the \POWHEGBOX{}
generator uses the Born cross section, rather than the NLO $\bar{B}$
function, to generate the underlying Born configuration. We can see that, in
this case, the LHE result is much closer to the NLO result, except for small
transverse momenta, where Sudakov damping becomes manifest. This results
confirms that the enhancement of the transverse-momentum distribution of the
second hardest jet is indeed due to the $\bar{B}/B$ $K$-factor.


\begin{figure}[htb]
\begin{center}
\includegraphics[height=\hfigs]{plots/HJ_j2-pt_cuts_fixsc}\nolinebreak
\includegraphics[height=\hfigs]{plots/HJ_j2-pt_cuts_runsc}
\caption{Transverse momentum distribution of the second hardest jet for
  different cuts on the first jet transverse momentum.}
\label{fig:HJ_j2-pt_cuts}
\end{center}
\end{figure}
In fig.~\ref{fig:HJ_j2-pt_cuts} we show the second hardest jet transverse
momentum for different cuts on the first jet $\pT$. From the figure it is
clear that, as long as the second jet $\pT$ is below the first jet $\pT$ cut,
the behaviour of the cross section is roughly as an inverse power of the
second jet transverse momentum.  By raising the transverse momentum above the
first jet $\pT$ cut, also the first jet is forced to have larger transverse
momentum, and the cross section falls much more rapidly. The comparison among
the LHE, PY and NLO curves shows the typical pattern, with the NLO diverging
at small transverse momenta, the LHE being instead suppressed in that region,
but raising above the NLO result because of the $\bar{B}/B$ $K$-factor.


\begin{figure}[htb]
\begin{center}
\includegraphics[height=\hfig]{plots/HJ_ptrel1_fixsc}\nolinebreak
\includegraphics[height=\hfig]{plots/HJ_ptrel1_runsc}
\caption{The relative transverse momenta of the partons within the hardest
  jet $\ptrelone$ as defined in eq.~(\ref{eq:ptrel1}).}
\label{fig:HJ_ptrel1}
\end{center}
\end{figure}
In fig.~\ref{fig:HJ_ptrel1} we plot the relative transverse momenta of the
partons within the hardest jet $\ptrelone$.  This quantity is defined as
follows: we perform a longitudinal boost to a frame where the rapidity of the
first jet $j_1$ is zero. In this frame we compute
\begin{equation}
\label{eq:ptrel1}
p_{\sss T}^{ \sss {\rm rel},j_1}=\sum_{i\in j_1}
\frac{|\vec{p}_i\times \vec{k}_j|}{|\vec{k}_j|}\,,
\end{equation}
where $k_j$ is the momentum of the first jet in this frame, and $p_i$ the
momenta of the partons clustered within the first jet.  This quantity, when
computed at the LHE level, displays a marked difference from the
corresponding NLO result. As discussed in ref.~\cite{Alioli:2010xa}, this can
be easily understood if we remember that the LHE result is suppressed by a
Sudakov form factor that requires that no hardest radiation has been emitted
from either the initial- or the final-state partons.  The large bin at
$\ptrelone=0$ in the LHE result is due to events where there is only one
parton in the jet, that are most likely ISR events. On the other hand, the
hardest radiation provided by \PYTHIA{} should restore a shape closer to the
NLO result, although amplified by the $\bar{B}/B$ factor.  Further
amplification of the showered result is due to the fact that the shower uses
a running coupling evaluated at the scale of the radiation, while the NLO
result uses higher scales, and to the presence of multiple emissions in the
shower.  It is clear that this distribution, being determined mostly by the
shower program, is quite sensitive to the shower model and tuning, and to the
interface between the shower and \POWHEGBOX{}.

\clearpage
\subsection{Results for $Hjj$ production}
We have run the $Hjj$ program for two scale choices, $\muf=\mur=\mH$
and $\muf=\mur=\HThat$. All the following plots will come in pairs,
the left one referring to the first scale choice, and the right one referring to
the second scale choice.

\begin{figure}[htb]
\begin{center}
\includegraphics[height=\hfig]{plots/HJJ_mult_fixsc} \nolinebreak
\includegraphics[height=\hfig]{plots/HJJ_mult_runsc} 
\caption{Two and three jet multiplicities in $Hjj$ production. In the left plot
  the scale is chosen equal to the mass of the Higgs. In the right plot the
  scale is taken equal to $\HThat$.}
\label{fig:njjmult}
\end{center}
\end{figure}
We begin by showing in fig.~\ref{fig:njjmult} the two and three jet
multiplicities in the $Hjj$ process.  We find considerable agreement of the
LHE, PY and NLO output for both the two and three jet multiplicities for the
left plot, which corresponds to the choice of scale $\muf=\mur=\mH$. The
right plot corresponds to the $\HThat$ choice of scale. In this plot the two jet
multiplicity agrees for all cuts, while the three-jet multiplicity displays
marked differences between the LHE result and the NLO, and for the $100$ GeV
transverse momentum cut, also between the fully showered result and the
NLO. We now explain the reason for these differences: with the $\mH$
scale choice the $K$-factor is near one, and thus the $\bar{B}/B$
amplification that usually enhances the spectrum of the radiated jet (that in
this case is the third jet) is not effective. The $\HThat$ scale is
considerably larger than $\mH$, especially with a large $\pT$ cut, which
implies a reduction of the Born cross section and an increase of the
$K$-factor.

\begin{figure}[htb]
\begin{center}
\includegraphics[height=\hfigs]{plots/HJJ_H-pt_fixsc} \nolinebreak
\includegraphics[height=\hfigs]{plots/HJJ_H-pt_runsc} 
\caption{Transverse-momentum distribution of the Higgs boson in a $Hjj$
  sample with a 20~GeV $\pT$ cut on the two hardest jets. The left and right
  plots use respectively the $\mH$ and $\HThat$ scales.}
\label{fig:HJJ_H-pt}
\end{center}
\end{figure}
In fig.~\ref{fig:HJJ_H-pt} we show the transverse-momentum distribution of
the Higgs boson in a $Hjj$ sample with a 20~GeV $\pT$ cut on the two hardest
jets.  Good agreement is found between the LHE, PY and NLO curves, except for
small Higgs transverse momenta, where differences of the order of 30\% are
found. This is not surprising, in view of the required presence of the two
relatively soft jets. Radiation off these jets is Sudakov suppressed in the
LHE result with respect to the NLO one.  Since this radiation would deplete
the jets, the LHE result suffers less depletion, and is thus larger. On the
other hand, the shower degrades the energy of the jets lowering the cross
section, an effect visible in the PY result. At large Higgs boson $\pT$, at
least one of the jets is forced to be hard, and thus these effects lose
importance.

\begin{figure}[htb]
\begin{center}
\includegraphics[height=\hfigs]{plots/HJJ_H-y_fixsc}  \nolinebreak
\includegraphics[height=\hfigs]{plots/HJJ_H-y_runsc} 
\caption{Higgs boson rapidity distribution in $Hjj$ for $\pT$ cuts of $20$,
  $50$~ and $100$~GeV on the hardest jets. The left and right plots use
  respectively the $\mH$ and $\HThat$ scales.  In the lower pane, the ratio
  of the LHE and PY results with respect to the NLO one, for the $20$~GeV
  cut.}
\label{fig:HJJ_H-y}
\end{center}
\end{figure}
In fig.~\ref{fig:HJJ_H-y} we show the Higgs boson rapidity distribution in
$Hjj$ for $\pT$ cuts of $20$, $50$ and $100$~GeV on the two hardest
jets. Again, since this is an inclusive distribution, it displays good
agreement between the LHE, PY and NLO results. The ratio is only displayed
for the $20\;$GeV cut. The three predictions become even more consistent for
larger $\pT$ cuts, as one would expect.


\begin{figure}[htb]
\begin{center}
\includegraphics[height=\hfigs]{plots/HJJ_j1-pt_fixsc}  \nolinebreak
\includegraphics[height=\hfigs]{plots/HJJ_j1-pt_runsc} 
\caption{Transverse-momentum distribution of the hardest jet in the $Hjj$
  process. The left and right plots use respectively the $\mH$ and $\HThat$
  scales.}
\label{fig:HJJ_j1-pt}
\end{center}
\end{figure}

\begin{figure}[htb]
\begin{center}
\includegraphics[height=\hfigs]{plots/HJJ_j2-pt_fixsc}   \nolinebreak
\includegraphics[height=\hfigs]{plots/HJJ_j2-pt_runsc} 
\caption{Transverse-momentum distribution of the second hardest jet in the
  $Hjj$ process. The left and right plots use respectively the $\mH$ and
  $\HThat$ scales.}
\label{fig:HJJ_j2-pt}
\end{center}
\end{figure}
In figs.~\ref{fig:HJJ_j1-pt} and~\ref{fig:HJJ_j2-pt} we display the
transverse-momentum distribution for the first and second jet. The first
distribution bears some similarity to the Higgs transverse momentum, for
inclusiveness reasons. Observe that the small momentum disagreement between
the LHE and the NLO result observed for the Higgs transverse momentum case is
less evident in the hardest jet plot, and even less evident in the second jet
one. This is explained by the fact that the same point in the abscissa
corresponds to an increasing hardness of the event in the Higgs, first jet
and second jet $\pT$ distributions.

\begin{figure}[htb]
\begin{center}
\includegraphics[height=\hfigs]{plots/HJJ_j3-pt_fixsc}  \nolinebreak
\includegraphics[height=\hfigs]{plots/HJJ_j3-pt_runsc} 
\caption{Transverse-momentum distribution of the third hardest jet in the
  $Hjj$ process. The left and right plots use respectively the $\mH$ and
  $\HThat$ scales.}
\label{fig:HJJ_j3-pt}
\end{center}
\end{figure}
Turning to less inclusive quantities, we show in fig.~\ref{fig:HJJ_j3-pt} the
transverse-momentum distribution of the third hardest jet.  We recognize here
the typical behaviour of the LHE results, with the damping of the
small-momentum growth found in the NLO result, and with the typical increase
due to the $\bar{B}/B$ factor at higher momenta.  As already anticipated in
the discussion on the jet multiplicity, we notice that the increase is
modest for the $\mH$ choice of scale, but is very large for the
$\HThat$ scale.

\begin{figure}[htb]
\begin{center}
\includegraphics[height=\hfigs]{plots/HJJ_j3-pt_cuts_fixsc}\nolinebreak  
\includegraphics[height=\hfigs]{plots/HJJ_j3-pt_cuts_runsc} 
\caption{Transverse-momentum distribution of the third hardest jet in the
  $Hjj$ process for the transverse momentum cut on the two hardest jets equal
  to $20$, $50$ and $100$~GeV.}
\label{fig:HJJ_j3-pt_cuts}
\end{center}
\end{figure}
In fig.~\ref{fig:HJJ_j3-pt_cuts} we display the third-jet transverse momentum
using three different cuts on the $\pT$ of the first and second jet. The
pattern is similar to what was already observed in the $Hj$ case. In the left
plot, where the scale is chosen equal to $\mH$, we see a better concordance
of the LHE and NLO results as the transverse momentum increases.

\begin{figure}[htb]
\begin{center}
\includegraphics[height=\hfigs]{plots/HJJ_ptrel1_fixsc} \nolinebreak
\includegraphics[height=\hfigs]{plots/HJJ_ptrel1_runsc} 
\caption{$\ptrelone$ distribution of the hardest jet in the $Hjj$
  process. The left and right plots use respectively the $\mH$ and $\HThat$
  scales.}
\label{fig:HJJ_ptrel1}
\end{center}
\end{figure}

\begin{figure}[htb]
\begin{center}
\includegraphics[height=\hfigs]{plots/HJJ_ptrel2_fixsc}  \nolinebreak
\includegraphics[height=\hfigs]{plots/HJJ_ptrel2_runsc} 
\caption{$\ptreltwo$ distribution of the second hardest jet in the $Hjj$
  process. The left and right plots use respectively the $\mH$ and $\HThat$
  scales.}
\label{fig:HJJ_ptrel2}
\end{center}
\end{figure}
In figs.~\ref{fig:HJJ_ptrel1} and~\ref{fig:HJJ_ptrel2} we show the $\ptrel$
distribution for the hardest and second hardest jet respectively. Here too
the pattern is similar to what already observed for the $Hj$ case. The LHE
distribution is strongly suppressed, due to the fact that small
$\ptrel$ values also imply that no initial state radiation has taken place
above that scale, and the distribution is dominated by the shower effects,
that are not obtained with the same scale choice as the NLO result. We can
again notice that also the $\bar{B}/B$ $K$-factor plays a role here,
especially for the right plots, that have a larger $K$-factor because of the
use of the $\HThat$ scale.



\clearpage
\subsection{Hadronization effects and matching ambiguities}
In this section we focus upon two topics: the effects of hadronization and
the ambiguities related to matching the LHE result to the shower.  In
figs.~\ref{fig:HJ_j2-pt_cuts_had} and~\ref{fig:HJJ_j3-pt_had_cuts} we display
the transverse momentum distribution of the radiated jet for different cuts
on the first jet $\pT$, for the $Hj$ and $Hjj$ generators respectively.
\begin{figure}[htb]
\begin{center}
\includegraphics[height=\hfigs]{plots/HJ_j2-pt_cuts_had_fixsc} \nolinebreak
\includegraphics[height=\hfigs]{plots/HJ_j2-pt_cuts_had_runsc}
\caption{Transverse-momentum distribution of the second hardest jet in the
  $Hj$ process for the $\pT$ cut on the hardest jet equal to
  $20$, $50$ and $100$~GeV.}
\label{fig:HJ_j2-pt_cuts_had}
\end{center}
\end{figure}

\begin{figure}[htb]
\begin{center}
\includegraphics[height=\hfigs]{plots/HJJ_j3-pt_cuts_had_fixsc}  \nolinebreak
\includegraphics[height=\hfigs]{plots/HJJ_j3-pt_cuts_had_runsc} 
\caption{Transverse-momentum distribution of the third hardest jet in the
  $Hjj$ process for the $\pT$ cut on the two hardest jets equal
  to $20$, $50$ and $100$~GeV.}
\label{fig:HJJ_j3-pt_had_cuts}
\end{center}
\end{figure}
The curves labelled ``PYscalup'' are obtained with a non-default
determination of the {\tt scalup} parameter to be set in the Les Houches
interface, which limits the hardness of the radiation of the following
shower. While normally in \POWHEG{} the {\tt scalup} is set to the hardest
momentum of the radiation that \POWHEG{} generates, in the ``PYscalup''
results, we determine it by finding the smallest transverse momentum of the
LHE, which can be either the transverse momentum of any parton relative to
the beam axis, or of any parton relative to any other parton, computed in the
center-of-mass frame. Because of the way the real-radiation contribution is
separated into singular contributions, the two choices differ only for
subleading configurations, and one expects only a minor effect due to this
change.  The plots in both figures confirm this expectation. Similarly, we
see that hadronization and underlying-event effects (indicated with HAD in
the figures) have a sizable impact on the distributions only for small
transverse momenta, as expected.

\subsection{Comparison between the $H$, $Hj$ and $Hjj$ generators}
In this section, we compare a few distributions that are described by more
than one available \POWHEGBOX{} generator. This comparison can be considered
as a first step in the direction of merging \POWHEGBOX{} samples with
increasing number of jets.


%% \begin{figure}[htb]
%% \begin{center}
%% \includegraphics[height=\hfigs]{plots/H-pt_comp_fixsc} \nolinebreak
%% \includegraphics[height=\hfigs]{plots/H-pt_comp_runsc} 
%% \caption{Comparison of the Higgs boson transverse-momentum distribution
%%   computed with the $H$, $Hj$ and \HqT{} generators. The left and right plots
%%   use respectively the $\mH$ and the Higgs $\pT$ scales in the $Hj$
%%   generator.}
%% \label{fig:H-pt_comp}
%% \end{center}
%% \end{figure}

\begin{figure}[htb]
\begin{center}
\includegraphics[height=0.5\textwidth]{plots/H-pt_comp_fixsc_runsc} 
\caption{Comparison of the Higgs boson transverse-momentum distribution
  computed with the $H$, $Hj$ and \HqT{} generators. The LHE $Hj$ results are
  shown for the $\mH$ and the Higgs underlying-Borm $\pT^{\sss\rm UB}$
  scale choice.}
\label{fig:H-pt_comp}
\end{center}
\end{figure}
We first consider a comparison between the Higgs boson transverse-momentum
spectrum obtained using the $H$ and the $Hj$ generators. The $H$ generator
describes this distribution in the whole $\pT$ range, with the Sudakov region
obtained with NLL accuracy, and with leading-order accuracy for the
high-momentum region.  On the other hand, the $Hj$ generator describes this
distribution with NLO accuracy, but only for transverse momenta above the
Sudakov region.

In these comparisons, the $H$ sample was obtained following the
recommendation of ref.~\cite{Dittmaier:2012vm}, i.e.~the parameter {\tt
  hfact} was set to $\mH/1.2$ in the {\tt powheg.input} file. With this
setup, the resulting Higgs boson $\pT$ distribution is in remarkable
agreement with the output of the \HqT{} program~\cite{hqt, Bozzi:2003jy,
  Bozzi:2005wk, deFlorian:2011xf}.  Furthermore, we apply a $K$-factor of
$1.32$ to the $H$ result, in order to match the \HqT{} total NNLO cross
section ($\sigma_{\sss H}^{\sss \rm NLO} = 10.85$~pb, $\sigma_{\sss H}^{\sss
  \rm NNLO} = 14.35$~pb). The results are displayed in
fig.~\ref{fig:H-pt_comp}.  The two figures differ only for the scale choice
used in the $Hj$ generator, since \HqT{} accepts only a fixed renormalization
and factorization scale.  As one can see, the agreement of the rescaled $H$
and \HqT{} generators is excellent, as remarked in
ref.~\cite{Dittmaier:2012vm}. The $Hj$ generator, using as scale the Higgs
boson mass, is also in excellent agreement with \HqT{} for large transverse
momenta. This is not surprising, since, in this kinematic region, it is using
the same ${\cal O}(\as^4)$ result of \HqT{}, and furthermore it is using the
same factorization and renormalization scale.  At low $\pT$, the $Hj$ results
begin to feel the lack of soft-gluon resummation effects that are included in
\HqT{}.
%% In the right plot we see a substantially similar pattern, with the $Hj$
%% result undershooting the \HqT{} one by $20\div 25$\%{}, a difference that is
%% easily explained as being due to the different scale adopted in the $Hj$
%% program and in \HqT{}.
The $Hj$ result computed with the dynamical scale shows a substantially
similar pattern, undershooting the \HqT{} one by $20\div 25$\%, a difference
that is easily explained as being due to the different scale adopted in the
$Hj$ program and in \HqT{}.


\begin{figure}[htb]
\begin{center}
\includegraphics[height=\hfigs]{plots/j2-pt_comp_fixsc} \nolinebreak
\includegraphics[height=\hfigs]{plots/j2-pt_comp_runsc} 
\caption{Comparison of the second hardest jet transverse momentum computed
  with the $Hj$ and $Hjj$ generators. The left plot uses the $\mH$ scale in
  both the $Hj$ and $Hjj$ generators. In the right plot the $Hj$ generator
  uses the Higgs underlying-Born transverse momentum scale, while the $Hjj$
  generator uses $\HThat$.}
\label{fig:j2-pt_comp}
\end{center}
\end{figure}
Turning to the comparison of the $Hj$ and $Hjj$ generators for the
transverse-momentum distribution of the second hardest jet, we notice that
the $Hj$ generator computes this distribution down to small values of the
transverse momentum, since it includes the appropriate Sudakov effects.  At
large transverse momenta, however, it only has leading-order accuracy.  The
$Hjj$ program has instead NLO accuracy at large transverse momenta, but does
not include fully resummed Sudakov effects at small $\pT$.  We do not have
any higher-accuracy calculation for this distribution, as in the previous
case. The results are displayed in fig.~\ref{fig:j2-pt_comp}.  Notice that,
for the $\mH$ scale choice, the matching is very good, something that we
expect since the $K$-factor is close to one in this case. Matching with the
running scales leads, instead, to non-negligible differences, although a
stability plateau is present roughly above $60$~GeV, and for not too large
values of the transverse momentum.

\section{Conclusions}
\label{sec:conc}
In the present work we have developed generators for the gluon fusion
production of a Higgs boson in association with one or two jets. We have
examined the output of the two generators, comparing them with fixed
order NLO calculations, among each other, and with the fully inclusive
$gg\to H$ \POWHEG{} generator.
The features of the distributions we have examined
are all well understood, and reflect our expectations for a typical \POWHEG{}
generator. In general, we find that quantities inclusive in the radiated
jet (i.e. the second jet in $gg\to H+1\,$jet and the third jet in
 $gg\to H+2\,$jets) are in good agreement with the fixed order NLO result,
while ditributions in the radiated jet reflect the features of Sudakov
suppression and NLO enhancement that are typical of NLO results matched with
shower generators. We see no indication of problems related to the increase
in complexity when going from the inclusive Higgs production to the
associated production with one and two jets, other than an increased
requirement of computer time.

The development of the Higgs production code has been achieved using
a new interface to the \MG{} code, that has also been presented in this
work. The use of this interface has considerably simplified the
construction of the generator. The interface is fully generic,
and so we expect that the implementation of new processes will be greatly
simplified with its use.

The code of our new Higgs production generators can be accessed
via the \POWHEGBOX{} svn repository see the \POWHEG{} web page
\url{http://powhegbox.mib.infn.it}
for instructions). The new \MG{} interface is also available there,
so that people willing to develop their own \POWHEG{}
code may benefit from its use.


%\section{Acknowledgments}



\appendix
\section{The interface with \MG: technical details}
\label{app:tech_details}
The \MG{} interface is available under the {\tt POWHEG-BOX/\MGS} directory.
To use the interface, create a process directory in the {\tt POWHEG-BOX}
directory, and copy the whole content of the \MGS\ directory to this new
process directory. To specify the process, edit the file
\verb|Cards/proc_card.dat| and set the process and the physics model as in
\MG. Always enter the real emission process, i.e.~the Born process plus an
extra jet \verb|j|. For example, to generate Higgs boson plus two jets at a
proton-proton collider, we entered
\begin{verbatim}
   p p > H j j j
   QCD=3
   QED=0
   HIG=1
\end{verbatim}
It is recommended to set the parameters \verb|QCD| and \verb|QED| to be
exactly equal to the number of strong and electroweak interactions in the
real-emission process (excluding the couplings present in the effective
vertex, if any). In the case of the \verb|heft| model (see below), it is also
needed to set the parameter \verb|HIG=1| to allow for the effective Higgs
boson to gluon coupling to be included in the Feynman diagrams.

The ordering of the particles in the process should follow the \POWHEGBOX{}
conventions: 
\begin{enumerate}
  \item first particle: incoming particle with positive rapidity
  
  \item second particle: incoming particle with negative rapidity
  
  \item from the third particle onward: final-state particles ordered as
  follows
  \begin{itemize}
    \item colourless particles first,
    
    \item massive coloured particles,
    
    \item massless coloured particles.
  \end{itemize}
\end{enumerate}
%
The default \MG\ interface has been validated to work with the
following three physics models:
\begin{itemize}
\item \verb|sm|: This is the default \MG\ model with a massive top
  and bottom quarks, diagonal CKM matrix and massless electrons and
  muons.
\item \verb|smckm|: This is the same as the \verb|sm| model, however,
  the full CKM matrix can be specified in the parameter card.
\item \verb|heft|: This is the same as the default \verb|sm| model with the
  inclusion of the effective coupling between gluons and the Higgs boson in
  the large top-quark mass limit. The Higgs Yukawa interaction with bottom
  quarks is neglected in this model.
\end{itemize}
Although the interface has not been tested with other physics models,
it is straight-forward to extend the interface to work with any simple
extension of the Standard Model.

To generate the process, execute the \verb|NewProcess.sh| script. This will
compile the \MG\ code, generate the (correlated) Born and real-emission
squared amplitudes and create the libraries that can be linked to the
\POWHEGBOX. In particular, \verb|libmadgraph.a| contains the matrix elements,
\verb|libdhelas3.a| the \HELAS{} library and \verb|libmodel.a| the physics
model. Furthermore, the following files are written in the current directory:
\begin{itemize}
\item \verb|Born.f|: it contains the routines \verb|setborn| to compute
  the (correlated) Born squared matrix elements and
  \verb|borncolour_lh| that assigns the colour flow to a Born process.
\item \verb|real.f|: it contains the routines \verb|setreal| to compute
  the real-emission squared matrix elements.
%  and \verb|realcolour_lh|
%  that assigns a colour flow to a real emission phase-space point.
\item \verb|Cards/param_card.dat|: this is the input card where all the model
  parameters need to be specified.
\item \verb|init_couplings.f|: this contains the \verb|init_couplings|
  routine that sets all the model parameters specified in the
  \verb|param_card.dat|. The coupling constants that depend on the strong
  coupling \verb|st_alpha| or from the event kinematics 
%such as $g_{\sss s}$ or the Higgs boson effective  coupling to gluons $g_H$, 
should be specified in the routine \verb|set_ebe_couplings|, that is updated
event-by-event.
\item \verb|coupl.inc|: it contains the common blocks for all the couplings
  used by \MG.
\end{itemize}
To complete the implementation of a process in the \POWHEGBOX{}, the user
must provide the Born phase-space in the file \verb|Born_phsp.f| and the
virtual squared matrix elements in the file \verb|virtual.f|. Also, no
information on possible intermediate resonances in the matrix elements is
kept. The user needs to specify explicitly in the routine \verb|finalize_lh|
(in the \verb|Born.f| file) which resonances should be written in the LHE
file, so that the shower Monte Carlo program can deal with them correctly.
Finally, the resulting code can be compiled  by executing the command
\begin{verbatim}
   $ make pwhg_main
\end{verbatim}


\section{\PYTHIA{} setup}
\label{app:PY_setup}
The sequence of \PYTHIA{} calls we have used in the calculation of the
results presented in sec.~\ref{sec:phenomenology} is the following:
\begin{itemize}
\item  without hadronization
\begin{verbatim}
         call PYINIT('USER','','',0d0)
c Hadronization off
         mstp(111)=0
c primordial kt off
         mstp(91)=0
c No multiple parton interactions
         if(mstp(81).eq.1) then
c Q2 ordered shower
            mstp(81)=0
         elseif(mstp(81).eq.21) then
c p_t^2 ordered shower
            mstp(81)=20
         endif
         call PYABEG
         call PYEVNT
         call PYANAL
\end{verbatim}

\item with hadronization 
\begin{verbatim}
         call PYTUNE(320)
         call PYINIT('USER','','',0d0)
c switching off MPI
         mstp(81)=20
         call PYABEG
         call PYEVNT
         call PYANAL
\end{verbatim}
\end{itemize}

\bibliography{paper}


\end{document}




