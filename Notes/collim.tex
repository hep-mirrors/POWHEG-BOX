\documentclass{letter}
\usepackage{amsmath}

%%%%%%%%%% Start TeXmacs macros
\newcommand{\section}[1]{\medskip\bigskip

\noindent\textbf{\LARGE #1}}
\newcommand{\tmop}[1]{\ensuremath{\operatorname{#1}}}
%%%%%%%%%% End TeXmacs macros

\begin{document}

\title{Collinear limit of tree amplitudes}\author{}\maketitle

\section{The $g \rightarrow g$ process}

Consider the $g \rightarrow g g$ process. An incoming gluon with momentum $p$
and Lorentz index $\mu$ enters the vertex, a gluon with momentum $k$ and index
$\nu$ leaves the vertex and enters the amplitude, a gluon with momentum $p -
k$ leaves the vertex. We set
\begin{equation}
  k = x p + \eta \xi + k_T, p^2 = 0, (p - k)^2 = k_T^2 - 2 (1 - x) \xi \eta
  \cdot p = 0, \xi = \frac{k_T^2}{2 \eta \cdot p (1 - x)} .
\end{equation}
We also need
\[ (p - k)^2 = - 2 p \cdot k + k^2 = 0 \Longrightarrow k^2 = 2 p \cdot k = 2 p
   \cdot \eta \xi = \frac{k_T^2}{1 - x} \]
The 3-gluon vertex is given by
\begin{equation}
  \Gamma^{\mu \nu \rho} (p, - k, k - p) = (p + k)^{\rho} g^{\mu \nu} + (p - 2
  k)^{\mu} g^{\nu \rho} + (k - 2 p)^{\nu} g^{\mu \rho},
\end{equation}
and using gauge invariance we can replace
\begin{eqnarray}
  (p + k)^{\rho} & \Rightarrow & (p + k)^{\rho} - \frac{1 + x}{1 - x} (p -
  k)^{\rho} = \frac{2}{1 - x} (\eta \xi + k_T)^{\rho} \nonumber\\
  (p - 2 k)^{\mu} & \Rightarrow & (p - 2 k)^{\mu} - (1 - 2 x) p^{\mu} = - 2
  (\eta \xi + k_T)^{\mu} \nonumber\\
  (k - 2 p)^{\nu} & \Rightarrow & (k - 2 p)^{\nu} + \frac{2 - x}{x} k^{\nu} =
  \frac{2}{x} (\eta \xi + k_T)^{\nu} 
\end{eqnarray}
so that the vertex can be replaced with
\begin{equation}
  \frac{2}{1 - x} k_T^{\rho} g^{\mu \nu} - 2 k_T^{\mu} g^{\nu \rho} +
  \frac{2}{x} k_T^{\nu} g^{\mu \rho} .
\end{equation}
This leads to terms of order $k_T^2$ when squared. Had we kept $\xi$ terms we
would have had non singular terms at the end. We now square the vertex, to get
\begin{eqnarray}
  \frac{4}{(1 - x)^2} k_T^2 g^{\nu \nu'}_T + 4 k_T^2 g^{\nu \nu'}_T +
  \frac{8}{x^2} k_T^{\nu} k_T^{\nu'} + \left[ - \frac{1}{1 - x} - \frac{1}{x}
  + \frac{1}{x (1 - x)} \right] 4 k_T^{\nu} k_T^{\nu'} & = &  \nonumber\\
  4 k_T^2 \frac{\frac{x}{1 - x} + \frac{1 - x}{x} + x (1 - x)}{x (1 - x)}
  g^{\nu \nu'} + \frac{8}{x^2} \left[ k_T^{\nu} k_T^{\nu'} - \frac{g^{\nu
  \nu'}}{2} k_T^2 \right] &  & 
\end{eqnarray}
We now divide by $k^4$, using also $k_T^2 = (1 - x) k^2$, and get the AP
factor
\begin{equation}
  C_A \frac{1}{x}  \frac{- 1}{k^2}  \left\{ - 4 \left[ \frac{x}{1 - x} +
  \frac{1 - x}{x} + x (1 - x) \right] g^{\nu \nu'} + \frac{8 (1 - x)}{x}
  \left[ \hat{k}_T^{\nu}  \hat{k}_T^{\nu'} + \frac{g^{\nu \nu'}}{2} \right]
  \right\}
\end{equation}
where $\hat{k}_T = k_T / \sqrt{|k_T^2 |}$. $C_A$ originates from the colour
sum. Colour and spin averages do not make a difference in this case, since a
gluon enters the amplitude as well as the subamplitude. The $1 / x$ factor
turns consistently the amplitude and subamplitudes squared into cross
sections.

Notice that $- g^{\nu \nu'} A^{(r)}_{\nu \nu'}$ is positive, and $- 1 / k^2$
is also positive.

\section{The $q \rightarrow g$ process}

Consider now the case $q \rightarrow q g$, with the gluon entering the
subamplitude. We should compute
\begin{equation}
  \tmop{tr} (p, \nu, p - k, \nu') = 4 (p^{\nu} (p - k)^{\nu'} + p^{\nu'} (p -
  k)^{\nu} - p \cdot (p - k) g^{\nu \nu'}) .
\end{equation}
Using gauge invariance we get rid of $k^{\nu}$ and $k^{\nu'}$, and get
\begin{equation}
  4 (2 p^{\nu} p^{\nu'} + p \cdot k g^{\nu \nu'}) .
\end{equation}
Now we substitute
\begin{equation}
  p^{\nu} \rightarrow p^{\nu} - \frac{1}{x} k \approx - \frac{1}{x} k_T,
\end{equation}
and get
\begin{eqnarray}
  4 \left( \frac{2}{.x^2} k_T^{\nu} k_T^{\nu'} + p \cdot k g^{\nu \nu'}
  \right) = 4 p \cdot k \left( \frac{4 (1 - x)}{x^2}  \frac{k_T^{\nu}
  k_T^{\nu'}}{k_T^2} + g^{\nu \nu'} \right) & = &  \nonumber\\
  4 p \cdot k \left( \frac{4 (1 - x)}{x^2} \left[ \frac{k_T^{\nu}
  k_T^{\nu'}}{k_T^2} - \frac{g^{\nu \nu'}}{2} \right] + g^{\nu \nu'} \frac{1 +
  (1 - x)^2}{x^2} \right), &  & 
\end{eqnarray}
and dividing by $k^4$ we get
\begin{equation}
  T_F \frac{1}{x}  \frac{- 1}{k^2} \left( - g^{\nu \nu'} \frac{1 + (1 -
  x)^2}{x} + \frac{4 (1 - x)}{x} \left[ \hat{k}_T^{\nu}  \hat{k}_T^{\nu'} +
  \frac{g^{\nu \nu'}}{2} \right] \right) .
\end{equation}
Again, spin average makes no difference. For the color, colour averaging
yields $1 / N_c$, so $T_F / N_c = C_F / D_A$, where $D_A$ is the dimension of
the adjoint representation, that provides the colour average for the
subamplitude.

\section{Crossing}

Invariant amplitude satisfy crossing relations. The collinear factorization
formula
\begin{equation}
  A (p) = \frac{1}{x k^2} P (x, k_T) A_r (k),
\end{equation}
with $x = k \cdot q / p \cdot q$ should also hold if the particles with
momentum $p$ and $k$ are crossed to outgoing particles. In this case one
interprets $z = p \cdot q / k \cdot q$, so we should let $x = 1 / z$. For the
$g \rightarrow g$ process we get
\begin{eqnarray}
  C_A z \frac{- 1}{k^2}  \left\{ - 4 \left[ \frac{1}{z - 1} + z - 1 + \frac{z
  - 1}{z^2} \right] g^{\nu \nu'} + 8 (z - 1) \left[ \hat{k}_T^{\nu} 
  \hat{k}_T^{\nu'} + \frac{g^{\nu \nu'}}{2} \right] \right\} & = & 
  \nonumber\\
  C_A  \frac{1}{k^2}  \left\{ - 4 \left[ \frac{z}{1 - z} + z (1 - z) + \frac{1
  - z}{z} \right] g^{\nu \nu'} + 8 z (1 - z) \left[ \hat{k}_T^{\nu} 
  \hat{k}_T^{\nu'} + \frac{g^{\nu \nu'}}{2} \right] \right\} &  & 
\end{eqnarray}
For the $g \rightarrow q \bar{q}$ process, remembering to include a - sign for
crossing a fermion line, we get
\begin{eqnarray*}
  T_F z \frac{1}{k^2} \left( - g^{\nu \nu'} (z + z (1 - 1 / z)^2) + 4 (z - 1)
  \left[ \hat{k}_T^{\nu}  \hat{k}_T^{\nu'} + \frac{g^{\nu \nu'}}{2} \right]
  \right) & = & \\
  T_F  \frac{1}{k^2} \left( - g^{\nu \nu'} \left( z^2 + (1 - z)^2 \right) - 4
  z (1 - z) \left[ \hat{k}_T^{\nu}  \hat{k}_T^{\nu'} + \frac{g^{\nu \nu'}}{2}
  \right] \right) &  & 
\end{eqnarray*}

\end{document}
