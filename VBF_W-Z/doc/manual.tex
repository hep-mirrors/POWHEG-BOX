\documentclass[a4paper,11pt]{article}
\usepackage{amssymb,enumerate}
\usepackage{amsmath}
\usepackage{bbm}
\usepackage{url}
\usepackage{cite}
\usepackage{graphics}
\usepackage{xspace}
\usepackage{epsfig}
\usepackage{subfigure}

\setlength\paperwidth  {210mm}%
\setlength\paperheight {300mm}%	

\textwidth 160mm%		% DEFAULT FOR LATEX209 IS a4
\textheight 230mm%

\voffset -1in
\topmargin   .05\paperheight	% FROM TOP OF PAGE TO TOP OF HEADING (0=1inch)
\headheight  .02\paperheight	% HEIGHT OF HEADING BOX.
\headsep     .03\paperheight	% VERT. SPACE BETWEEN HEAD AND TEXT.
\footskip    .07\paperheight	% FROM END OF TEX TO BASE OF FOOTER. (40pt)


\hoffset -1in				% TO ADJUST WITH PAPER:
	\oddsidemargin .13\paperwidth	% LEFT MARGIN FOR ODD PAGES (10)
	\evensidemargin .15\paperwidth	% LEFT MARGIN FOR EVEN PAGES (30)
	\marginparwidth .10\paperwidth	% TEXTWIDTH OF MARGINALNOTES
	\reversemarginpar		% BECAUSE OF TITLEPAGE.

\newcommand{\tmtextit}[1]{{\itshape{#1}}}
\newcommand{\tmtexttt}[1]{{\ttfamily{#1}}}
\newenvironment{enumeratenumeric}{\begin{enumerate}[ 1.] }{\end{enumerate}}
\newcommand\sss{\mathchoice%
{\displaystyle}%
{\scriptstyle}%
{\scriptscriptstyle}%
{\scriptscriptstyle}%
}
\newcommand\PSn{\Phi_{n}}
\newcommand{\tmop}[1]{\ensuremath{\operatorname{#1}}}

\newcommand\Lum{{\cal L}}
\newcommand\matR{{\cal R}}
\newcommand\Kinnpo{{\bf \Phi}_{n+1}}
\newcommand\Kinn{{\bf \Phi}_n}
\newcommand\PSnpo{\Phi_{n+1}}
\newcommand\as{\alpha_{\sss\rm S}}
\newcommand\asotpi{\frac{\as}{2\pi}}

\newcommand\POWHEG{{\tt POWHEG}}
\newcommand\POWHEGBOX{{\tt POWHEG\;BOX}}
\newcommand\PYTHIA{{\tt PYTHIA}}
\newcommand\POWHEGpPYTHIA{{\tt POWHEG+PYTHIA}}
\newcommand\HERWIG{{\tt HERWIG}}

\title{Manual for $W$ and $Z$ Production via Vector Boson Fusion in the \POWHEGBOX{}}
\date{}
%\author{}
%\keywords{}
%\abstract{}
%\preprint{}


\begin{document}
\maketitle
%
\noindent
This document describes the settings and input parameters that are specific to
the implementation of $W$ and $Z$ production via vector boson fusion at the LHC within the
\POWHEGBOX{} framework. 
%
The parameters that are common to all \POWHEGBOX{} implementations are given in
the {\tt manual-BOX.pdf} document in the {\tt POWHEG-BOX/Docs}
directory. Since there are singular configurations in the Born contributions, due care has to be exercised 
to avoid these regions when generating the Born phase space. Two available options are described in 
the Input parameters section. On this account, the code is not expected to give 
reliable results when used with very inclusive cuts.
%\\[2ex]
If you use our program, please cite
Refs.~\cite{SZ,Oleari:2003tc,Alioli:2010xd}.

\section*{Running the program}
Go to the directory
{\tt \$ cd POWHEG-BOX/VBF\_W-Z}
and create a directory where you want to run the code:
\\[2ex]
{\tt \$ mkdir testrun}
\\[2ex]
The directory must contain the {\tt powheg.input} and {\tt vbfnlo.input} file.
To compile the code with {\tt gfortran}, type 
\\[2ex]
{\tt \$ make}
\\[2ex]
The code needs {\tt fastjet} and {\tt lhapdf}, so make sure that {\tt fastjet-config} and 
{\tt lhapdf-config} are in the path.%
\\[2ex]
After compiling, enter the testrun directory:
\\[2ex]
{\tt \$ cd testrun}
\\[2ex]
The program can be run sequentially by deactivating the variable {\tt manyseeds} in {\tt powheg.input}. Execute the program with
\\[2ex]
{\tt \$../pwhg\_main}
\\[2ex]
The program can also be run in parallel by setting {\tt manyseeds} to {\tt 1} in {\tt powheg.input}
and in several steps as for instance described in the manual 
for VBF Z production \cite{JSZ}.
\\[2ex]
\section*{Input parameters}
To have more flexible choices for the input parameters of the calculation, a new input file 
({\tt vbfnlo.input}) was added 
in addition to the standard {\tt powheg.input} file. The use of this new file and some special parameters 
for {\tt powheg.input} will be explained in the following.
\subsection*{{\tt powheg.input}}

\begin{itemize}
\item {\tt mll\_gencut }: cut on the invariant dilepton mass for $Zjj$ production, set to 15 GeV if lower than 15 GeV. This 
cut is needed to avoid singularities when a virtual photon decays into two massless leptons.
\item  {\tt bornsuppfact}: \begin{itemize}
        \item[\textbf{0}]  No Born suppression factor. You should use generation cuts via {\tt  ptj\_gencut} when using this option.  
        \item[\textbf{1}]  Use a Born suppression factor $F(\Phi_n)$ with 
        $$F(\Phi_n) = \left( \frac{p_{T,j1}^2}{p_{T,j1}^2+\Lambda_{pTj}^2} \right)^n \left( \frac{p_{T,j2}^2}{p_{T,j2}^2+\Lambda_{pTj}^2} \right)^n.$$
        $\Lambda_{pTj}$ is set to the value given in  {\tt ptsuppfact} and $F(\Phi_n)$ vanishes whenever 
        a singular region in the Born phase space $\Phi_n$ is reached. The underlying Born kinematics are then generated 
in the \textsc{Powheg-Box}  according to a modified $\overline{B}$ function,
$$\overline{B}_{\text{supp}}=\overline{B}(\Phi_n) F(\Phi_n).$$
     \end{itemize}
\item {\tt ptj\_gencut}: Generation cut on the jets' $p_T$ in the phase space generator. Should be used when  {\tt bornsuppfact} 
is set to 0.
\item {\tt Phasespace}: \begin{itemize}
        \item[\textbf{1}] Default value, use the standard phase space (with {\tt bornsuppfact} or {\tt ptj\_gencut}).  
        \item[\textbf{2}] Use unweighted events as phase space generator.  To 
use this option, unweighted events with very inclusive cuts have to be generated with \textsc{Vbfnlo} in a seperate run. Each 
event which survives the unweighting procedure has to be reweighted by the factor $J_i=\sigma_{LO}/(|\mathcal{M}_B(\Phi_i)|^2 \,\text{pdf}(\Phi_i))$, 
the Born cross section over the respective numerical value of the squared Born matrix element including pdfs. 
This factor $J_i$ is exactly the Jacobian factor of the Born phase space. Place the file with the reweighted events called 
{\tt event.total.lhe} in the running directory.
        \end{itemize}
\item {\tt bornxsec}: Born cross section in nb of the unweighted events used for {\tt Phasespace} 2. Not needed otherwise.
\item {\tt fakevirt }: If the program is run in several steps, the generation of the grid can be 
performed with {\tt fakevirt 1}, where the virtual amplitude is set to a factor proportional to the Born amplitude.
\end{itemize}

\subsection*{{\tt vbfnlo.input}}
The input parameters which can be changed by the user are explained in the 
\textsc{Vbfnlo} Manual \cite{vbfnlo} in detail. Here we give a list of the supported parameters 
with a short description.

\begin{itemize}
\item {\tt PROC\_ID}: \begin{itemize}
        \item[\bf 120]  $p p \to Z \, jj \to \ell^{+} \ell^{-} \, jj$  
        \item[\bf 130]  $p p \to W^{+} \,  jj\to \ell^{+} \nu_\ell \, jj$  
        \item[\bf 140]  $p p \to W^{-} \, jj\to \ell^{-} \bar{\nu}_\ell  \, jj$  \end{itemize} 
\item {\tt DECAYMODE}: \begin{itemize}
        \item[\bf 11]   $\ell = e$ (electron)
        \item[\bf 13]  $\ell = \mu$ (muon)   \end{itemize}         
\item {\tt HMASS}: Standard Model Higgs boson mass in GeV. Default value is 126~GeV.
\item {\tt HWIDTH}: It is possible to set the Higgs boson width with this input
parameter.  Default is -999~GeV, which means that the internally calculated
value of the width is used. 
\item {\tt EWSCHEME}: Sets the scheme for the calculation of electroweak
parameters. A summary of the four available options is given in
Table~\ref{tab:Schemes}. Note that if {\tt EWSCHEME = 4} is
chosen, all variables in Table~\ref{tab:Schemes} are taken as inputs.  As the
parameters are not independent, this can lead to problems if the input values
are not consistent.  In this scheme, all photon couplings are set according to
the input variable {\tt INVALFA} and all other couplings are set according to
{\tt FERMI\_CONST}. Default value is 3.
\item {\tt ANOM\_CPL}: If set to {\tt 1}, anomalous gauge
boson couplings are used if available for the selected process. Anomalous
coupling parameters are set via the file {\tt anomV.dat} which has to be located 
in the running directory.
For details on the possible choices and parameterisations of 
the anomalous couplings see the \textsc{Vbfnlo} Manual \cite{vbfnlo}. This is 
not a tested feature and should always be compared to the NLO result of 
\textsc{Vbfnlo}. Default value is {\tt 0}. 
\item {\tt TOPMASS}: Top-quark mass in GeV. Default value is 172.4~GeV. 
\item {\tt TAU\_MASS}: Tau mass in GeV used in the calculation of the Higgs boson
width and branching ratios. Default value is 1.77684~GeV. 
\item {\tt BOTTOMMASS}: Bottom-quark pole mass in GeV, used in the calculation
of the Higgs boson width and branching ratios. Default value is
4.855~GeV, which corresponds to $m_{b}^{\overline{MS}}(m_{b}) = 4.204$~GeV.
\item {\tt CHARMMASS}: Charm-quark pole mass in GeV used in the calculation of
the Higgs boson width and branching ratios. Default value is 1.65~GeV,
corresponding to $m_{c}^{\overline{MS}}(m_{c}) = 1.273$~GeV.
\item {\tt FERMI\_CONST}: Fermi constant, used as input for the calculation of
electroweak parameters in {\tt EWSCHEME} = 1-4. Default value is $1.16637 \times
10^{-5} \ \mathrm{GeV}^{-2}$. 
\item {\tt INVALFA:} One over the fine structure constant, used as input for
{\tt EWSCHEME} $=$ 1 and 4.  Within the other schemes this parameter is
calculated. The default value depends on the choice of {\tt EWSCHEME}, as given
in Table~\ref{tab:Schemes}.
\end{itemize} 
To compare the code with \cite{JSZ}, the flag {\tt consistency} in the {\tt qqbqq.f} and 
{\tt qqZWjqqi.f} file in the {\tt vbfnlo-files} directory has to be set to {\tt .true.}. 
Than, all amplitudes with $Q^2<4 \text{~GeV}$ are provided with a large damping factor, not 
only the ones with photons in the $t$-channel. After that, the code has to be recompiled. 


\begin{table}[t!]
\newcommand{\lstrut}{{$\strut\atop\strut$}}
\begin{center}
\begin{tabular}{c|c|l|c}
\hline
&&&\\
{\tt EWSCHEME} & \textsc{Parameter} & \textsc{Default Value} & \textsc{Input/Calculated} \\
&&&\\
\hline
&&&\\
 & {\tt FERMI\_CONST} & $1.16637\times 10^{-5} \ \mathrm{GeV}^{-2}$ & \textsc{Input}\\
 & {\tt INVALFA} & $128.944341122$ & \textsc{Input}\\
\bf 1 & {\tt SIN2W} & $0.230990$ & \textsc{Calculated} \\
 & {\tt WMASS} & $79.9654 \ \mathrm{GeV}$ & \textsc{Calculated}\\
 & {\tt ZMASS} & $91.1876 \ \mathrm{GeV}$ & \textsc{Input}\\
&&&\\
\hline
&&&\\
 & {\tt FERMI\_CONST} & $1.16637 \times 10^{-5} \ \mathrm{GeV}^{-2}$ & \textsc{Input}\\
 & {\tt INVALFA} & $132.340643024$ & \textsc{Calculated}\\
\bf 2 & {\tt SIN2W} & $0.222646$ & \textsc{Input}\\
 & {\tt WMASS} & $ 80.3980 \ \mathrm{GeV}$ & \textsc{Calculated}\\
 & {\tt ZMASS} & $91.1876 \ \mathrm{GeV}$ & \textsc{Input}\\
&&&\\
\hline
&&&\\
 &{\tt FERMI\_CONST} & $1.16637 \times 10^{-5} \ \mathrm{GeV}^{-2}$ & \textsc{Input}\\
 &{\tt INVALFA} & $132.340705199$ & \textsc{Calculated}\\
\bf 3 &{\tt SIN2W} & $0.222646$ & \textsc{Calculated}\\
 &{\tt WMASS} & $80.3980 \ \mathrm{GeV}$ & \textsc{Input}\\
 & {\tt ZMASS} & $91.1876 \ \mathrm{GeV}$ & \textsc{Input}\\
&&&\\
\hline
&&&\\
 &{\tt  FERMI\_CONST} & $1.16637 \times 10^{-5} \ \mathrm{GeV}^{-2}$ & \textsc{Input}\\
 &{\tt  INVALFA} & $137.035999679$ & \textsc{Input}\\
\bf 4 &{\tt  SIN2W} & $0.222646$ & \textsc{Input}\\
 &{\tt  WMASS} & $80.3980 \ \mathrm{GeV}$ & \textsc{Input}\\
 &{\tt  ZMASS} & $91.1876 \ \mathrm{GeV}$ & \textsc{Input}\\
&&&\\
\hline
&&&\\
\end{tabular}
\caption {\em  Electroweak input parameter schemes.}
\vspace{0.2cm}
\label{tab:Schemes}
\end{center}
\end{table}

%===================================================================


% \subsection*{{\tt anomV.dat} $-$ anomalous triple gauge
%   boson couplings}
% 
% The anomalous triple gauge boson couplings can be set in the file
% {\tt anomV.dat}.  They are parameterized using an effective Lagrangian, as
% described in Refs.~\cite{Buchmuller:1985jz, Hagiwara:1993ck, Eboli:2006wa}
% \begin{equation}
%  \mathcal{L}_{\rm eff} = \frac{f_{i}}{\Lambda^{n}} \mathcal{O}_{i}^{n+4},
% \end{equation}
% where $n+4$ signifies the dimension of the operator $\mathcal{O}_{i}$. 
% \textsc{Vbfnlo} defines the anomalous gauge couplings in terms of the
% coefficients $f_{i}/\Lambda^{n}$ of the dimension-6 and dimension-8 
% operators. 
% 
% A common alternative parameterization (which \textsc{Vbfnlo} can also use as
% input) of the trilinear couplings $WW\gamma$ and $WWZ$ uses the following
% effective Lagrangians:
% \begin{equation}
% \begin{split}
% \label{anovertex}
% \mathcal{L}_{{{WW}}\gamma} =& -ie \left[ W_{\mu\nu}^\dagger W^\mu
% A^\nu- W_\mu^\dagger A_\nu W^{\mu\nu} +\kappa_\gamma W_\mu^\dagger
% W_\nu F^{\mu\nu} +{\lambda_\gamma\over m_W^2} W_{\sigma\mu}^\dagger
% W^\mu_\nu F^{\nu\sigma}\right]\,
% \end{split}
% \end{equation}
% for the anomalous $WW\gamma$ vertex, and
% \begin{equation}
% \begin{split}
% \label{anovertexZ}
% \mathcal{L}_{{{WWZ}}} = & -ie \cot \theta_w \,\left[ g_1^Z\left(
%   W_{\mu\nu}^\dagger W^\mu Z^\nu- W_\mu^\dagger Z_\nu
%   W^{\mu\nu}\right) +\kappa_Z W_\mu^\dagger W_\nu Z^{\mu\nu}
% +{\lambda_Z\over m_W^2} W_{\sigma\mu}^\dagger W^\mu_\nu
% Z^{\nu\sigma}\right]\,
% \end{split}
% \end{equation}
% for the anomalous $WWZ$ vertex. It is customary to rephrase the
% electroweak modifications around the SM Lagrangian in terms of new
% quantities,
% %
% \begin{equation}
% (\Delta g_1^Z,\Delta \kappa_Z, \Delta\kappa_\gamma) = ( g_1^Z,\kappa_Z,\kappa_\gamma) - 1 .
% \end{equation}
% 
% 
% In order to include anomalous vector boson couplings, the parameter {\tt
% ANOM\_CPL} must be switched to {\tt 1} in {\tt vbfnlo.input}.  The parameters
% described above are input via the file {\tt anomV.dat}:
% 
% \begin{itemize}
%  \item {\tt TRIANOM:} Switches between parameterizations of the anomalous
% $WW\gamma$ and $WWZ$ couplings.  {\tt TRIANOM = 1} uses the coefficients of the
% dimension-6 operators as input: 
%    \begin{itemize}
%      \item {\tt FWWW:} Coefficient of the operator $\mathcal{O}_{WWW}$, i.e.\
% $f_{WWW}$/$\Lambda^{2}$.  Default is set to $9.19~\times~10^{-6}$~GeV$^{-2}$.
%      \item {\tt FW:} Coefficient of the operator $\mathcal{O}_{W}$, i.e.\
% $f_{W}$/$\Lambda^{2}$.  Default is set to $-1.44~\times~10^{-5}$~GeV$^{-2}$.
%      \item {\tt FB:} Coefficient of the operator $\mathcal{O}_{B}$, i.e.\
% $f_{B}$/$\Lambda^{2}$.  Default is set to $3.83~\times~10^{-5}$~GeV$^{-2}$.
%    \end{itemize}
%   {\tt TRIANOM = 2} uses the alternative parameterization of Eqs.~\ref{anovertex}~and~\ref{anovertexZ} as input:
%    \begin{itemize}
%     \item {\tt LAMBDA0:} The quantity $\lambda_{\gamma} (= \lambda_{Z})$. 
% Default is set to 0.038.
%     \item {\tt ZDELTAKAPPA0:} The quantity $\Delta \kappa_{Z}$.  Default is set
% to -0.082.
%     \item {\tt ZDELTAG0:} The quantity $\Delta g_{1}^{Z}$.  Default is set to -0.060.
%     \item {\tt ADELTAKAPPA0:} The quantity $\Delta \kappa_{\gamma}$.  Default
% is set to 0.077.
%     \item  Note that, as can be seen from Eqs.~\ref{eq:param2}, the quantities $\Delta
% \kappa_{Z}$, $\Delta \kappa_{\gamma}$ and $\Delta g_{1}^{Z}$ are not
% independent, but obey the relation
%  \begin{equation}
%   \label{eq:anomVconsistency}
%   \Delta \kappa_{Z} = \Delta g_{1}^{Z} - \frac{\sin^{2} \theta_{W}}{\cos^{2} \theta_{W}} \Delta \kappa_{\gamma}.
%  \end{equation}
%  If one of these quantities is zero, it will be set by \textsc{Vbfnlo} to be
% consistent with the other values.  If the input values are inconsistent, $\Delta
% \kappa_{\gamma}$ will be reset to give the correct relation.
%    \end{itemize}
% 
%   Default is {\tt TRIANOM = 1}, and the default values are approximately the central values in the three-parameter fit given in Ref.~\cite{Abdallah:2010zj}.  Note that $Vjj$ and $VVj$ processes (process IDs 120-150 and 610-640) only take account of the above anomalous  coupling parameters and not the following parameters.
%  \item {\tt FWW, FBB:} The coefficients of the remaining $\cal{CP}$-even
% dimension-6 operators, i.e.\ $f_{i}$/$\Lambda^{2}$.  Note that these are not
% implemented for $Vjj$ production via VBF (process IDs 120-150) and $VVj$
% processes (IDs 610-640).  Default value is 0 GeV$^{-2}$.
%  \item {\tt FWWWt, FWt, FBt, FBWt, FDWt, FWWt, FBBt:} The coefficients of the
% $\cal{CP}$-odd di\-mension-6 operators, i.e.\ $f_{i}$/$\Lambda^{2}$.  Note that
% these are
% not implemented for $Vjj$ production via VBF (process IDs 120-150) and $VVj$
% processes (IDs 610-640).  Default value is 0  GeV$^{-2}$.
%  \item {\tt  FS0, FS1, FM0 -- FM7, FT0 -- FT2, FT5 -- FT9:} Parameters that give
% the values of the coefficients of the dimension-8 operators, i.e.\
% $f_{i}$/$\Lambda^{4}$. The default values for these parameters are~0 GeV$^{-4}$.  Note that
% these are only relevant for triboson production and $VVjj$ production via VBF.
% \end{itemize}
% 
% In addition, a form factor can be applied, which takes the form
% \begin{equation}
%  F = \left(1 + \frac{s}{\Lambda^{2}} \right)^{-p},
% \end{equation}
% for all processes except $\gamma jj$ production in VBF (process ID 150),
% where $\Lambda$ is the scale of new physics.  $s$ is a universal scale (the
% invariant mass squared of the produced bosons) for each phase-space point.
% \begin{itemize}
%  \item {\tt FORMFAC: }  Switch determining whether the above form factor $F$ is
% included.  Default is set to {\tt true}.
%  \item {\tt FFMASSSCALE: } Mass scale $\Lambda$.  Default is set to 2000~GeV.
%  \item {\tt FFEXP:} The exponent $p$.  Default is set to 2.
% \end{itemize}
% Individual form factor mass scales and exponents can be set for the trilinear
% $WWZ$ and $WW\gamma$ couplings, using either of the parametrizations above.  If
% chosen, these values overwrite the ``universal'' mass scale and exponent ({\tt
% FFMASSSCALE} and {\tt FFEXP}) set above for the selected parameters.
% \begin{itemize}
%  \item {\tt FORMFAC\_IND: } Switch determining whether individual or universal
% form factors are used.  Default is set to {\tt false} -- universal form factors are used.
% \end{itemize}
% If {\tt TRIANOM = 1} then
% \begin{itemize}
%  \item {\tt MASS\_SCALE\_FWWW: } Mass scale $\Lambda$ for coefficient $f_{WWW}$.
%  Default is 2000~GeV.
%  \item {\tt FFEXP\_FWWW: } Exponent $p$ for coefficient $f_{WWW}$.  Default is 2.
%  \item {\tt MASS\_SCALE\_FW: } Mass scale $\Lambda$ for coefficient $f_{W}$. 
% Default is 2000~GeV.
%  \item {\tt FFEXP\_FW: } Exponent $p$ for coefficient $f_{W}$.  Default is 2.
%  \item {\tt MASS\_SCALE\_FB: } Mass scale $\Lambda$ for coefficient $f_{B}$. 
% Default is 2000~GeV.
%  \item {\tt FFEXP\_FB: } Exponent $p$ for coefficient $f_{B}$.  Default is 2.
% \end{itemize}
% If {\tt TRIANOM = 2} then
% \begin{itemize}
%  \item {\tt MASS\_SCALE\_AKAPPA: } Mass scale $\Lambda$ for parameter $\Delta
% \kappa_{\gamma}$.  Default is 2000~GeV.
%  \item {\tt FFEXP\_AKAPPA: } Exponent $p$ for parameter $\Delta \kappa_{\gamma}$.  Default
% is 2.
%  \item {\tt MASS\_SCALE\_ZKAPPA: } Mass scale $\Lambda$ for parameter $\Delta
% \kappa_{Z}$.  Default is 2000~GeV.
%  \item {\tt FFEXP\_ZKAPPA: } Exponent $p$ for parameter $\Delta \kappa_{Z}$.  Default
% is 2.
%  \item {\tt MASS\_SCALE\_LAMBDA: } Mass scale $\Lambda$ for parameter $\lambda$.
%  Default is 2000~GeV.
%  \item {\tt FFEXP\_LAMBDA: } Exponent $p$ for parameter $\lambda$.  Default is
% 2.
%  \item {\tt MASS\_SCALE\_G: } Mass scale $\Lambda$ for parameter $\Delta g_{1}^{Z}$. 
% Default is 2000~GeV.
%  \item {\tt FFEXP\_G: } Exponent $p$ for parameter $\Delta g_{1}^{Z}$.  Default is 2.
%  \item As with the anomalous parameters themselves, the formfactors in the
% parameterization of {\tt TRIANOM = 2} are not independent (see
% Eq.~\ref{eq:anomVconsistency}).  If one mass scale is set to zero, it will be
% set by \textsc{Vbfnlo} to be consistent with the other values.  If the input
% values are inconsistent, they will be reset to give the correct relation.
% \end{itemize}



%
\begin{thebibliography}{99}

\bibitem{SZ} F.~Schissler, and D.~Zeppenfeld, {\em Parton Shower Effects on W and Z Production
 via Vector Boson Fusion at NLO QCD}, arXiv:1302.2884. 

\bibitem{Oleari:2003tc} 
  C.~Oleari and D.~Zeppenfeld,
  {\em QCD corrections to electroweak nu(l) j j and l+ l- j j production},
  Phys.\ Rev.\ D {\bf 69}, 093004 (2004)
  [hep-ph/0310156].
\bibitem{Alioli:2010xd} S.~Alioli, P.~Nason, C.~Oleari, and E. Re, {\em
    A general framework for implementing NLO calculations in shower
    Monte Carlo programs: the POWHEG BOX}, JHEP {\bf 1006} (2010)
  043  [arXiv:1002.2581 [hep-ph]].
  
  \bibitem{JSZ} B.~J\"ager, S.~Schneider, G.~Zanderighi, {\em
    Next-to-leading order QCD corrections to electroweak $Zjj$ 
    production in the POWHEG BOX}, arXiv:1207.2626 [hep-ph].
\bibitem{vbfnlo}   
  K.~Arnold, M.~B{\"a}hr, G.~Bozzi, F.~Campanario, C.~Englert, T.~Figy, N.~Greiner and C.~Hackstein {\it et al.},
  %``VBFNLO: A Parton level Monte Carlo for processes with electroweak bosons,''
  Comput.\ Phys.\ Commun.\  {\bf 180}, 1661 (2009)
  [arXiv:0811.4559 [hep-ph]];\\
  %%CITATION = ARXIV:0811.4559;%%
  K.~Arnold, J.~Bellm, G.~Bozzi, F.~Campanario, C.~Englert, B.~Feigl, J.~Frank and T.~Figy {\it et al.},
  %``Release Note -- Vbfnlo-2.6.0,''
  [arXiv:1207.4975 [hep-ph]];\\
  %%CITATION = ARXIV:1207.4975;%%
  K.~Arnold, J.~Bellm, G.~Bozzi, M.~Brieg, F.~Campanario, C.~Englert, B.~Feigl and J.~Frank {\it et al.},
  %``VBFNLO: A Parton Level Monte Carlo for Processes with Electroweak Bosons -- Manual for Version 2.5.0,''
  [arXiv:1107.4038 [hep-ph]].
  %%CITATION = ARXIV:1107.4038;%%
% \bibitem{Buchmuller:1985jz}
% W.~Buchmuller and D.~Wyler, ``{Effective Lagrangian Analysis of New Interactions
%   and Flavor Conservation}'', {\em Nucl.\ Phys.} {\bf B268} (1986)
% 621.

% \bibitem{Eboli:2006wa}
% O.~J.~P.~Eboli, M.~C.~Gonzalez-Garcia and J.~K.~Mizukoshi, 
%                   ``{$p p \to j j e^\pm \mu^\pm \nu \nu$ and $j j e^\pm \mu^\mp \nu \nu$ at
%                   $O(\alpha_{\rm em}^6)$ and $O(\alpha_{\rm em}^4 \alpha_S^2)$ for the
%                   study of the quartic electroweak gauge boson vertex at
%                   LHC}'', {\em Phys.\ Rev.} {\bf D74} (2006) 073005,
% {{\tt hep-ph/0606118}}.
% 
% \bibitem{Abdallah:2010zj}
%   The DELPHI Collaboration,
%   ``{Measurements of CP-conserving Trilinear Gauge Boson Couplings $WWV$ ($V =
%     \gamma,Z$) in $e^+e^-$ Collisions at LEP2}'',
%   {\em Eur.\ Phys.\ J.}  {\bf C66 } (2010)  35-56,
%   {{\tt arXiv:1002.0752}}.
\end{thebibliography}
%%%%%%%%%%%%%%%%%%
\end{document}
