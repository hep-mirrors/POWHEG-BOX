\documentclass[paper]{JHEP3}
\usepackage{amsmath,amssymb,enumerate,url}
\bibliographystyle{JHEP}

%%%%%%%%%% Start TeXmacs macros
\newcommand{\tmtextit}[1]{{\itshape{#1}}}
\newcommand{\tmtexttt}[1]{{\ttfamily{#1}}}
\newenvironment{enumeratenumeric}{\begin{enumerate}[1.] }{\end{enumerate}}
\newcommand\sss{\mathchoice%
{\displaystyle}%
{\scriptstyle}%
{\scriptscriptstyle}%
{\scriptscriptstyle}%
}

\newcommand\as{\alpha_{\sss\rm S}}
\newcommand\POWHEG{{\tt POWHEG}}
\newcommand\POWHEGBOX{{\tt POWHEG BOX}}
\newcommand\HERWIG{{\tt HERWIG}}
\newcommand\PYTHIA{{\tt PYTHIA}}
\newcommand\MCatNLO{{\tt MC@NLO}}

\newcommand\kt{k_{\sss\rm T}}

\newcommand\pt{p_{\sss\rm T}}
\newcommand\LambdaQCD{\Lambda_{\scriptscriptstyle QCD}}
%%%%%%%%%% End TeXmacs macros


\title{The POWHEG BOX user manual:\\
  $\boldsymbol{W^\pm}$ production} \vfill
\author{Simone Alioli\\
  Deutsches Elektronen-Synchrotron DESY\\
  Platanenallee 6, D-15738 Zeuthen, Germany\\
  E-mail: \email{simone.alioli@desy.de}}
\author{Paolo Nason\\
  INFN, Sezione di Milano-Bicocca,
  Piazza della Scienza 3, 20126 Milan, Italy\\
  E-mail: \email{Paolo.Nason@mib.infn.it}}
\author{Carlo Oleari\\
  Universit\`a di Milano-Bicocca and INFN, Sezione di Milano-Bicocca\\
  Piazza della Scienza 3, 20126 Milan, Italy\\
  E-mail: \email{Carlo.Oleari@mib.infn.it}}
\author{Emanuele Re\\
  Institute for Particle Physics Phenomenology, Department of Physics\\
  University of Durham, Durham, DH1 3LE, UK\\
  E-mail: \email{emanuele.re@durham.ac.uk}}

\vskip -0.5truecm

\keywords{POWHEG, Shower Monte Carlo, NLO}

\abstract{This note documents the use of the package
  \tmtexttt{POWHEG BOX} for $W^\pm$ production processes.
  Results can be easily interfaced to shower Monte Carlo programs, in
  such a way that both NLO and shower accuracy are maintained.}
\preprint{\today\\ \tmtexttt{POWHEG BOX} 1.0}

\begin{document}


\section{Introduction}

The \tmtexttt{POWHEG BOX} program is a framework for implementing NLO
calculations in Shower Monte Carlo programs according to the
\POWHEG{} method. An explanation of the method and a discussion of
how the code is organized can be found in
refs.~\cite{Nason:2004rx,Frixione:2007vw,Alioli:2010xd}.  The code is
distributed according to the ``MCNET GUIDELINES for Event Generator
Authors and Users'' and can be found at the web page \\
%
\begin{center}
 \url{http://powhegbox.mib.infn.it}.
\end{center}
%
~\\
%

This program is an implementation of the Drell-Yan NLO cross sections
$pp\to W \to \ell \nu $. A detailed description of the
implementation can be found in ref.~\cite{Alioli:2008gx}. Please cite
the paper when you use the program. The implementation included in the 
\tmtexttt{POWHEG-BOX/W} subdirectory package is based on the
subtraction scheme by Frixione, Kunszt and Signer rather than on the
scheme by Catani and Seymour discussed in the paper.

In order to run the \tmtexttt{POWHEG BOX} program, we recommend the
reader to start from the \tmtexttt{POWHEG BOX} user manual, which
contains all the information and settings that are common between
all subprocesses. In this note we focus on  the settings and
parameters specific to the $W$ implementation.

\section{Generation of events}

Build the executable\\
\tmtexttt{\$ cd POWHEG-BOX/W \\
\$ make pwhg\_main }\\
Then do (for example) \\
\tmtexttt{\$
cd testrun-wp-lhc\\
\$ ../pwhg\_main}\\
At
the end of the run, the file \tmtexttt{pwgevents.lhe} will contain
100000 events for $W^+ \to e^+ \nu_e$  in the Les Houches format. In
order to shower them with \PYTHIA{} do\\~\\
\tmtexttt{\$
cd POWHEG-BOX/W \\ \$ make
main-PYTHIA-lhef \\ \$ cd
testrun-wp-lhc \\
\$ ../main-PYTHIA-lhef}

\section{Process specific input parameters}

For $W^\pm$ production, it is required to activate the
\tmtexttt{withdamp} option, to enable the Born-zero damping factor.
Therefore, the line \\~\\ \tmtexttt{withdamp 1 ! (default 0, do not
  use) use Born-zero damping factor} \\~\\ should be added ( or
uncommented ) in the corresponding input files.

Other mandatory parameters are those needed to select the final state
leptonic species coming from the vector-boson:
~\\
\tmtexttt{
  idvecbos 24 \phantom{aaa} ! PDG code for vector boson to be produced\\
  \phantom{aaaaaaaaaaaaaaa} ! ( W+: 24 W-: -24 )\\
  vdecaymode 1 \phantom{aa} !  code for selected W decay\\
  \phantom{aaaaaaaaaaaaaaa} ! (1: electronic; 2: muonic; 3: tauonic)}
\\

Together with the mandatory parameters, the \POWHEGBOX{} input facility  allows for an easy setting of EW and run parameters, by explicitly adding the relevant lines to the input card. In case one of the following entries is not present in the input card  the reported default value is assumed. In any case, these parameters are printed in the output of the program, so their values can be easily tracked down. 
~\\~\\
 \tmtexttt{
   Wmass 80.398      ! W mass in GeV\\
   Wwidth 2.141      ! W width in GeV\\
   Zmass  91.1876    ! Z mass in GeV\\
   Zwidth 2.4952     ! Z width in GeV\\
   sthw2  0.22264    ! sin**2 theta\_w \\
   alphaem 0.0072973 ! em coupling }
\\
\tmtexttt{masswindow\_low 10   ! M\_W > Wmass - masswindow\_low * Wwidth}\\
\tmtexttt{masswindow\_high 10  ! M\_W < Wmass + masswindow\_high * Wwidth  }\\
\tmtexttt{runningscale 0  ! choice for ren and fac scales in Bbar integration}\\
\phantom{pppppppppppppppppppppppppppppp} \tmtexttt{ 0: fixed scale M\_W}\\
\phantom{pppppppppppppppppppppppppppppp} \tmtexttt{ 1: running scale inv mass W}\\ 
\phantom{pppppppppppppppppppppppppppppp} \tmtexttt{ 2: running scale transverse mass W}\\~\\~\\
\tmtexttt{
   CKM\_Vud   0.9740  ! Entries of CKM mixing matrix\\
   CKM\_Vus   0.2225  \\
   CKM\_Vub   0.000001 \\
   CKM\_Vcd   0.2225 \\
   CKM\_Vcs   0.9740 \\
   CKM\_Vcb   0.000001 \\
   CKM\_Vtd   0.000001 \\
   CKM\_Vts   0.000001 \\
   CKM\_Vtb   1.0\\}

\section{Generation of a sample with $W^+$ and $W^-$
  events}\label{sec:merging}

In case the user is interested in the generation of a sample where
both $W^+$ and $W^-$ events appear, a script and a dedicated
executable have been included. The script is named
\tmtexttt{merge\_wp\_wm.sh} and can be found in the directory
\tmtexttt{W/testrun-merge}. It can be run in any subfolder of
\tmtexttt{W} however. Three inputs are mandatory: the first two are
the prefixes of the input files used to generate $W^+$ and $W^-$
events. The third input has to be an integer and correspond to the
total number of events that the final \emph{merged} sample will
contain. The script has to be run twice, using a positive integer
value at the first call and its opposite afterwards.  Therefore, for
example, to produce a sample of 10000 events, starting from the input
files \tmtexttt{wp-powheg.input} and \tmtexttt{wm-powheg.input}, the
invocation lines should be as follows:~\\~\\ \tmtexttt{\$ sh
  merge\_wp\_wm.sh wp wm 10000}\\~\\ and then~\\~\\ \tmtexttt{\$ sh
  merge\_wp\_wm.sh wp wm -10000}\\~\\

Few remarks are needed:
\begin{itemize}
\item it is responsability of the user to check that the 2 input files
  are equal. The \tmtexttt{idvecbos}  and \tmtexttt{vdecaymode} tokens have to be different,
  obviously.
\item the two values of \tmtexttt{numevts} are not really used: the
  program re-calculate the needed values as a function of the $W^+$ and
  $W^-$ cross sections and of the total number of events to be
  generated.
\item the final event file is always named
  \tmtexttt{wp\_wm\_sample-events.lhe}. In the header section it also
  contains a copy of the two input files used to generate it, for
  cross-checking purposes
\end{itemize}


\begin{thebibliography}{10}

\bibitem{Nason:2004rx}
  P.~Nason,
  ``A new method for combining NLO QCD with shower Monte Carlo algorithms,''
  JHEP {\bf 0411} (2004) 040
  [arXiv:hep-ph/0409146].
  %%CITATION = JHEPA,0411,040;%%

%\cite{Frixione:2007vw}
\bibitem{Frixione:2007vw}
  S.~Frixione, P.~Nason and C.~Oleari,
``Matching NLO QCD computations with Parton Shower simulations: the POWHEG
method,''
  JHEP {\bf 0711} (2007) 070
  [arXiv:0709.2092 [hep-ph]].
  %%CITATION = JHEPA,0711,070;%%

%\cite{Alioli:2010xd}
\bibitem{Alioli:2010xd}
  S.~Alioli, P.~Nason, C.~Oleari and E.~Re,
``A general framework for implementing NLO calculations in shower Monte Carlo
  programs: the POWHEG BOX,''
  [arXiv:1002.2581 [hep-ph]].
  %%CITATION = ARXIV:1002.2581;%%

%\cite{Alioli:2008gx}
\bibitem{Alioli:2008gx} S.~Alioli, P.~Nason, C.~Oleari and E.~Re,
  ``NLO vector-boson production matched with shower in POWHEG,''
  JHEP {\bf 0807}, 060 (2008)
  [arXiv:0805.4802 [hep-ph]].


\end{thebibliography}

\end{document}





