\documentclass[paper]{JHEP3}
\usepackage{amsmath,amssymb,enumerate,url}
\bibliographystyle{JHEP}

%%%%%%%%%% Start TeXmacs macros
\newcommand{\tmtextit}[1]{{\itshape{#1}}}
\newcommand{\tmtexttt}[1]{{\ttfamily{#1}}}
\newenvironment{enumeratenumeric}{\begin{enumerate}[1.] }{\end{enumerate}}
\newcommand\sss{\mathchoice%
{\displaystyle}%
{\scriptstyle}%
{\scriptscriptstyle}%
{\scriptscriptstyle}%
}

\newcommand\as{\alpha_{\sss\rm S}}
\newcommand\POWHEG{{\tt POWHEG}}
\newcommand\POWHEGBOX{{\tt POWHEG BOX}}
\newcommand\HERWIG{{\tt HERWIG}}
\newcommand\PYTHIA{{\tt PYTHIA}}
\newcommand\PHOTOS{{\tt PHOTOS}}
\newcommand\MCatNLO{{\tt MC@NLO}}

\newcommand\kt{k_{\sss\rm T}}

\newcommand\pt{p_{\sss\rm T}}
\newcommand\LambdaQCD{\Lambda_{\scriptscriptstyle QCD}}
%%%%%%%%%% End TeXmacs macros


\title{The POWHEG BOX user manual:\\
  $\boldsymbol{W^\pm}$ NLO EW \& QCD production} \vfill
\author{Luca Barz\`e\\
  Dipartimento di Fisica Nucleare e Teorica, Universit\`a di Pavia and INFN, Sezione di Pavia, Via A. Bassi 6, 27100 Pavia, Italy\\
  E-mail: \email{Luca.Barze@pv.infn.it}}
\author{Guido Montagna\\
  Dipartimento di Fisica Nucleare e Teorica, Universit\`a di Pavia and INFN, Sezione di Pavia, Via A. Bassi 6, 27100 Pavia, Italy\\
  E-mail: \email{Guido.Montagna@pv.infn.it}}
\author{Paolo Nason\\
  INFN, Sezione di Milano-Bicocca,
  Piazza della Scienza 3, 20126 Milan, Italy\\
  E-mail: \email{Paolo.Nason@mib.infn.it}}
\author{Oreste Nicrosini\\
  INFN, Sezione di Pavia, Via A. Bassi 6, 27100 Pavia, Italy\\
  E-mail: \email{Oreste.Nicrosini@pv.infn.it}}
\author{Fulvio Piccinini\\
  INFN, Sezione di Pavia, Via A. Bassi 6, 27100 Pavia, Italy\\
  E-mail: \email{Fulvio.Piccinini@pv.infn.it}}

\vskip -0.5truecm

\keywords{POWHEG, Shower Monte Carlo, NLO, Electroweak}

\abstract{This note documents the use of the package
  \tmtexttt{POWHEG BOX} for $W^\pm$ production processes including
  QCD and ElectroWeak NLO corrections.
  Results can be easily interfaced to shower Monte Carlo programs, in
  such a way that both NLO and shower accuracy are maintained.}
\preprint{\today\\ \tmtexttt{POWHEG BOX} 1.0}

\begin{document}


\section{Introduction}

The \tmtexttt{POWHEG BOX} program is a framework for implementing NLO
calculations in Shower Monte Carlo programs according to the
\POWHEG{} method. An explanation of the method and a discussion of
how the code is organized can be found in
Refs.~\cite{Nason:2004rx,Frixione:2007vw,Alioli:2010xd}. The code is
distributed according to the ``MCNET GUIDELINES for Event Generator
Authors and Users'' and can be found at the web page \\
%
\begin{center}
 \url{http://powhegbox.mib.infn.it}.
\end{center}
%
~\\
%
This program is an implementation of the Drell-Yan NLO cross sections
$pp\to W \to \ell \nu$ including QCD and ElectroWeak (EW) 
radiative corrections. 
A detailed description of the implementation can be found in 
Ref.~\cite{Barze:2012tt}. Please cite the paper when you use the program. \\
In order to run the \tmtexttt{POWHEG BOX} program, we recommend the
reader to start from the \tmtexttt{POWHEG BOX} user manual, which
contains all the information and settings that are common between
all subprocesses. In this note we focus on  the settings and
parameters specific to the $W$ implementation.

\section{Generation of events}

Build the executable\\
\tmtexttt{\$ cd POWHEG-BOX/WEW \\
\$ make pwhg\_main }\\
Then do (for example) \\
\tmtexttt{\$
cd test-el\\
\$ ../pwhg\_main}\\
At
the end of the run, the file \tmtexttt{pwgevents.lhe} will contain
100000 events for $W^+ \to e^+ \nu_e$  in the Les Houches format. In
order to shower them with \PYTHIA{} do\\~\\
\tmtexttt{\$
cd POWHEG-BOX/WEW \\ \$ make
main-PYTHIA-lhef \\ \$ cd
test-el \\
\$ ../main-PYTHIA-lhef}

\section{Process specific input parameters}

For the EW NLO $W^\pm$ production, it is required to activate the
\tmtexttt{easlight} option, to enable the new final state radiation mapping
necessary to describe the final state collinear radiation.\\
Other mandatory parameters are those needed to select the final state
leptonic species coming from the vector-boson:
~\\
\tmtexttt{
  idvecbos 24 \phantom{a} ! PDG code for vector boson to be produced\\
  \phantom{aaaaaaaaaaaaa} ! ( W+: 24 W-: -24 )\\
  vdecaymode 11 !  code for selected W decay\\
  \phantom{aaaaaaaaaaaaa} ! (11(-11): electronic; 13(-13): muonic; 15(-15): tauonic)}
\\
Together with the mandatory parameters, the \POWHEGBOX{} input facility allows for an easy 
setting of EW and run parameters, by explicitly adding the relevant lines to the input card. 
If one of the following entries is not present in the input card  the reported default 
value is assumed. In any case, these parameters are printed in the output of the program, 
so their values can be easily tracked down. 
~\\~\\
 \tmtexttt{
   Wmass\phantom{aa}  80.398 \phantom{aaaaaaa} ! W mass in GeV\\
   Wwidth\phantom{a} 2.141 \phantom{aaaaaaaa} ! W width in GeV\\
   Zmass\phantom{aa}  91.1876 \phantom{aaaaaa}   ! Z mass in GeV\\
   Zwidth\phantom{a} 2.4952 \phantom{aaaaaaa} ! Z width in GeV\\
   alphaem 0.00729735254 \phantom{a}! em coupling alpha(0)\\
   Hmass\phantom{aa}  120. \phantom{aaaaaaaa} ! Higgs mass in GeV\\
   Tmass\phantom{aa} 172.9 \phantom{aaaaaaaa} ! Top mass in GeV\\
   gmu\phantom{aaaa} 1.16637d-5 \phantom{aaa} ! Fermi constant in GeV\^{}-2\\
}
\\
\tmtexttt{masswindow\_low 10 \phantom{a}! M\_W > Wmass - masswindow\_low * Wwidth}\\
\tmtexttt{masswindow\_high 10  ! M\_W < Wmass + masswindow\_high * Wwidth  }\\
\tmtexttt{runningscale 0 \phantom{aaaa}! choice for ren and fac scales in Bbar integration}\\
\phantom{pppppppppppppppppppppppppppppp} \tmtexttt{ 0: fixed scale M\_W}\\
\phantom{pppppppppppppppppppppppppppppp} \tmtexttt{ 1: running scale inv mass W}\\ 
\tmtexttt{scheme 1! choice for EW NLO scheme calculation}\\
\phantom{pppppppppppppppppppppppppppppp} \tmtexttt{ 0: Alpha(0)}\\
\phantom{pppppppppppppppppppppppppppppp} \tmtexttt{ 1: G\_$\mu$}\\ 
\tmtexttt{
   CKM\_Vud   0.975  ! Entries of CKM mixing matrix\\
   CKM\_Vus   0.222  \\
   CKM\_Vub   1d-10 \\
   CKM\_Vcd   0.222 \\
   CKM\_Vcs   0.975 \\
   CKM\_Vcb   1d-10 \\
   CKM\_Vtd   1d-10 \\
   CKM\_Vts   1d-10 \\
   CKM\_Vtb   1.0\\}
\\
\\
The EW radiative corrections can be calculated according to two 
different schemes: the $\alpha(0)$ scheme, where the input parameters 
are $\alpha(0)$, $M_W$ and $M_Z$; the $G_\mu$ scheme, where the 
input parameters are $G_\mu$, $M_W$ and $M_Z$. The latter one 
(default in the code) is preferred because it minimizes 
the EW corrections and the uncertainties due to the light quark masses. 
\\
As described in Ref.~\cite{Barze:2012tt}, higher order QED corrections 
are simulated by means of \PHOTOS~\cite{Golonka:2005pn} used in the 
exponentiated mode {\tt (IEXP=.TRUE.)}. 
An interface to the code is provided in the file {\tt main-PYTHIA.f} 
and the photon cascade information is stored in the {\tt common/hepeup}. 
\\
The \PHOTOS \, source code is included in the \POWHEGBOX. 



\section{Generation of a sample with $W^+$ and $W^-$
  events}\label{sec:merging}

In case the user is interested in the generation of a sample where
both $W^+$ and $W^-$ events appear, a script and a dedicated
executable have been included. The script is named
\tmtexttt{merge\_wp\_wm.sh} and can be found in the directory
\tmtexttt{W/testrun-merge}. It can be run in any subfolder of
\tmtexttt{W} however. Three inputs are mandatory: the first two are
the prefixes of the input files used to generate $W^+$ and $W^-$
events. The third input has to be an integer and correspond to the
total number of events that the final \emph{merged} sample will
contain. The script has to be run twice, using a positive integer
value at the first call and its opposite afterwards.  Therefore, for
example, to produce a sample of 10000 events, starting from the input
files \tmtexttt{wp-powheg.input} and \tmtexttt{wm-powheg.input}, the
invocation lines should be as follows:~\\~\\ \tmtexttt{\$ sh
  merge\_wp\_wm.sh wp wm 10000}\\~\\ and then~\\~\\ \tmtexttt{\$ sh
  merge\_wp\_wm.sh wp wm -10000}\\~\\

Few remarks are needed:
\begin{itemize}
\item it is responsability of the user to check that the 2 input files
  are equal. The \tmtexttt{idvecbos}  and \tmtexttt{vdecaymode} tokens have to be different,
  obviously.
\item the two values of \tmtexttt{numevts} are not really used: the
  program re-calculate the needed values as a function of the $W^+$ and
  $W^-$ cross sections and of the total number of events to be
  generated.
\item the final event file is always named
  \tmtexttt{wp\_wm\_sample-events.lhe}. In the header section it also
  contains a copy of the two input files used to generate it, for
  cross-checking purposes
\end{itemize}


\begin{thebibliography}{10}

\bibitem{Nason:2004rx}
  P.~Nason,
  ``A new method for combining NLO QCD with shower Monte Carlo algorithms,''
  JHEP {\bf 0411} (2004) 040
  [arXiv:hep-ph/0409146].
  %%CITATION = JHEPA,0411,040;%%

%\cite{Frixione:2007vw}
\bibitem{Frixione:2007vw}
  S.~Frixione, P.~Nason and C.~Oleari,
``Matching NLO QCD computations with Parton Shower simulations: the POWHEG
method,''
  JHEP {\bf 0711} (2007) 070
  [arXiv:0709.2092 [hep-ph]].
  %%CITATION = JHEPA,0711,070;%%

%\cite{Alioli:2010xd}
\bibitem{Alioli:2010xd}
  S.~Alioli, P.~Nason, C.~Oleari and E.~Re,
``A general framework for implementing NLO calculations in shower Monte Carlo
  programs: the POWHEG BOX,''
  [arXiv:1002.2581 [hep-ph]].
  %%CITATION = ARXIV:1002.2581;%%

\bibitem{Barze:2012tt}
  L.~Barz\`e, G.~Montagna, P.~Nason, O.~Nicrosini and F.~Piccinini,
  ``Implementation of electroweak corrections in the POWHEG BOX: single W production,''
  [arXiv:1202.0465 [hep-ph]]

%\cite{Golonka:2005pn}
\bibitem{Golonka:2005pn}
  P.~Golonka and Z.~Was,
  %``PHOTOS Monte Carlo: A Precision tool for QED corrections in $Z$ and $W$ decays,''
  Eur.\ Phys.\ J.\ C {\bf 45} (2006) 97
  [hep-ph/0506026].
  %%CITATION = HEP-PH/0506026;%%

\end{thebibliography}

\end{document}




