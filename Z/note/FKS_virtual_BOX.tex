\documentclass[11pt]{article} %ER: % Don't like 10pt? Try 11pt or 12pt

\usepackage[english]{inputenc}
\usepackage[english]{babel}
\usepackage[paper=a4paper]{geometry}

\usepackage{epsfig}
\usepackage{amsmath}
\usepackage{amssymb}
\usepackage{amsbsy}
\usepackage{bbm}
%\documentclass{article}
%\usepackage{amsmath,amssymb,enumerate,epsfig}

%%%%%%%%%% Start TeXmacs macros
%\newcommand{\tmmathbf}[1]{\ensuremath{\boldsymbol{#1}}}
\newcommand{\tmmathbf}[1]{\ensuremath{{\bf #1}}}
%\newcommand{\tmop}[1]{\ensuremath{\operatorname{#1}}}
%\newcommand{\tmtextit}[1]{{\itshape{#1}}}
\newcommand{\tmtexttt}[1]{{\ttfamily{#1}}}
\newcommand{\tmtextrm}[1]{{\rmfamily{#1}}}
%\newenvironment{enumeratenumeric}{\begin{enumerate}[1.] }{\end{enumerate}}
\newcommand{\tmop}[1]{\ensuremath{{\rm #1}}}
\newcommand{\tmtextit}[1]{{\itshape{#1}}}
\newenvironment{enumeratenumeric}{\begin{enumerate}}{\end{enumerate}}
\newenvironment{itemizedot}{\begin{itemize}
\renewcommand{\labelitemi}{$\bullet$}
\renewcommand{\labelitemii}{$\bullet$}
\renewcommand{\labelitemiii}{$\bullet$}
\renewcommand{\labelitemiv}{$\bullet$}}{\end{itemize}} 
%%%%%%%%%% End TeXmacs macros

\def\beq{\begin{equation}}
\def\beqn{\begin{eqnarray}}
\def\eeq{\end{equation}}
\def\eeqn{\end{eqnarray}}
\def\abs#1{\left|#1\right|}
\def\bentarrow{\:\raisebox{1.3ex}{\rlap{$\vert$}}\!\longrightarrow}
\def\FWeq#1{eq.~({\bf I}.#1)}
\def\FNWeq#1{eq.~({\bf II}.#1)}
\def\FKSeq#1{eq.~({\bf FKS}.#1)}
\def\half{\frac{1}{2}}

\newcommand\FKS{Frixione, Kunszt and Signer}
\newcommand\CS{Catani and Seymour}

\newcommand\HERWIG{{\tt HERWIG}}
\newcommand\HWpp{{\tt HERWIG}++}
\newcommand\PYTHIA{{\tt PYTHIA}}
\newcommand\ARIADNE{{\tt ARIADNE}}

\newcommand\BASES{{\tt BASES}}
\newcommand\SPRING{{\tt SPRING}}

\def\IN{{\small IN}}
\def\OUT{{\small OUT}}
\def\INm{{\sss\rm IN}}
\def\OUTm{{\sss\rm OUT}}

%\def\tk{\widetilde k}
\def\tk{\bar k}
\def\tx{\widetilde x}



\def\lq{\left[} 
\def\rq{\right]} 
\def\rg{\right\}} 
\def\lg{\left\{} 
\def\({\left(} 
\def\){\right)} 
 
\def\REF#1{#1}


\def\KFS#1{K^{{#1}}_{\scriptscriptstyle\rm F\!.S\!.}}
\def\HFS#1{H_{{#1}}^{\scriptscriptstyle\rm F\!.S\!.}}
\def\Kab{\KFS{ab}}
\def\Hba{\HFS{ba}}
\def\Kcb{\KFS{cb}}
\def\Kac{\KFS{ac}}
\def\Hac{\HFS{ac}}
\def\Hcb{\HFS{cb}}
\def\Hab{\HFS{ab}}
\def\cm{{\cal M}}
\def\bom#1{{\mbox{\boldmath $#1$}}}
\def\ket#1{|{#1}>}
\def\bra#1{<{#1}|}

\def\th{\theta}

%% initial-state labels

\newcommand\sss{\mathchoice%
{\displaystyle}%
{\scriptstyle}%
{\scriptscriptstyle}%
{\scriptscriptstyle}%
}

\newcommand\nplus{\oplus}
\newcommand\nminus{\ominus}

\newcommand\splus{{\sss \nplus}}
\newcommand\sminus{{\sss \nminus}}

\newcommand\splusminus{{\mathchoice%
{\vplusminus\displaystyle}%
{\vplusminus\scriptstyle}%
{\vplusminus\scriptscriptstyle}%
{\vplusminus\scriptscriptstyle}%
}}

\newcommand\sminusplus{{\mathchoice%
{\vminusplus\displaystyle}%
{\vminusplus\scriptstyle}%
{\vminusplus\scriptscriptstyle}%
{\vminusplus\scriptscriptstyle}%
}}

\newdimen\hbigcirc
\newdimen\wbigcirc

%\newcommand\vplusminus[1]{%
%\settoheight{\hbigcirc}{$#1\bigcirc$}%
%\settowidth{\wbigcirc}{$#1\bigcirc$}%
%\makebox[\wbigcirc]{\raisebox{0.1\hbigcirc}%
%{\makebox[0pt]{$#1+$}}\raisebox{-0.25\hbigcirc}%
%{\makebox[0pt]{$#1-$}}\makebox[0pt]{$#1\bigcirc$}}%
%}

\newcommand\vplusminus[1]{%
\settoheight{\hbigcirc}{$#1\bigcirc$}%
\settowidth{\wbigcirc}{$#1\bigcirc$}%
\makebox[\wbigcirc]{%
\makebox[0pt]{\rule[0.4\hbigcirc]{0.5\wbigcirc}{0.05\hbigcirc}}%
\makebox[0pt]{\rule[0.1\hbigcirc]{0.5\wbigcirc}{0.05\hbigcirc}}%
\makebox[0pt]{\rule[0.1\hbigcirc]{0.05\wbigcirc}{0.6\hbigcirc}}%
\makebox[0pt]{$#1\bigcirc$}}%
}


\newcommand\vminusplus[1]{%
\settoheight{\hbigcirc}{$#1\bigcirc$}%
\settowidth{\wbigcirc}{$#1\bigcirc$}%
\makebox[\wbigcirc]{%
\makebox[0pt]{\rule[0.2\hbigcirc]{0.5\wbigcirc}{0.05\hbigcirc}}%
\makebox[0pt]{\rule[0.5\hbigcirc]{0.5\wbigcirc}{0.05\hbigcirc}}%
\makebox[0pt]{\rule[-0.1\hbigcirc]{0.05\wbigcirc}{0.6\hbigcirc}}%
\makebox[0pt]{$#1\bigcirc$}}%
}

\newcommand\xplus{x_\splus}
\newcommand\xminus{x_\sminus}
\newcommand\xplusminus{x_\splusminus}
\newcommand\kplus{k_\splus}
\newcommand\kminus{k_\sminus}
\newcommand\Kplus{K_\splus}
\newcommand\Kminus{K_\sminus}
\newcommand\bxplus{\bar{x}_\splus}
\newcommand\bxminus{\bar{x}_\sminus}
\newcommand\bxplusminus{\bar{x}_\splusminus}
%

\newcommand\fmo{{f_{\splus}}}
\newcommand\fmop{{f_{\splus}'}}
\newcommand\fmt{{f_{\sminus}}}

\newcommand\ep{\epsilon}
\newcommand\clH{{\mathbb H}}
\newcommand\clS{{\mathbb S}}
\newcommand\EVprjmap{{\cal P}_{\clH\to\clS}}
\newcommand\as{\alpha_{\sss\rm S}}
\newcommand\asotpi{\frac{\as}{2\pi}}
\newcommand\Lum{{\cal L}}
\newcommand\Lumt{\tilde{\cal L}}
\newcommand\orda{{\cal O}(\as)}
\newcommand\ordaz{{\cal O}(\as^0)}
\newcommand\ordat{{\cal O}(\as^2)}
\newcommand\ordaB{{\cal O}(\as^m)}
\newcommand\ordaBpo{{\cal O}(\as^{m+1})}
\newcommand\pt{p_{\sss\rm T}}
\newcommand\ptmin{{\pt^{\min}}}
\newcommand\kt{k_{\sss\rm T}}
\newcommand\kperp{k_\perp}
\newcommand\tildept{\tilde{p}_{\sss\rm T}}
\newcommand\tildekt{\tilde{k}_{\sss\rm T}}
\newcommand\tildektmaxsq{{\tilde{k}_{\sss\rm T,\,max}^2}}
\newcommand\boost{\mathbb B}

%\newcommand\mat{{\cal M}}
%\newcommand\matB{{\cal M}^{(b)}}
%\newcommand\matR{{\cal M}^{(r)}}
%\newcommand\matRS{\widehat{\cal M}^{(r)}}
%\newcommand\matVb{{\cal M}^{(v)}_{\sss B}}
\newcommand\matB{{\cal B}}
\newcommand\matR{{\cal R}}
\newcommand\matRi{{\cal R}^{(\ctindex)}}
\newcommand\matRS{\widehat{\matR}}
\newcommand\matVb{{\cal V}_{\bare}}
\newcommand\matMCs{{\cal M}}
\newcommand\matI{{\cal I}}
%\newcommand\matSV{{\cal M}^{(sv)}}
\newcommand\matSV{{\cal V}}

%\newcommand\matCb{{\cal M}^{(c)}_{\sss B}}
%\newcommand\matCpb{{\cal M}^{(c+)}_{\sss B}}
%\newcommand\matCmb{{\cal M}^{(c-)}_{\sss B}}
%\newcommand\matCpmb{{\cal M}^{(c\pm)}_{\sss B}}
%\newcommand\matCpm{{\cal M}^{(c\pm)}}
\newcommand\matC{{\cal G}}
\newcommand\bare{{\rm b}}
\newcommand\matCb{{\cal G}_\bare}
\newcommand\matCpb{{\cal G}_{\splus,\bare}}
\newcommand\matCmb{{\cal G}_{\sminus,\bare}}
\newcommand\matCpmb{{\cal G}_{\splusminus,\bare}}
\newcommand\matCp{{\cal G}_\splus}
\newcommand\matCm{{\cal G}_\sminus }
\newcommand\matCpm{{\cal G}_{\splusminus}}

\newcommand\ctindex{\alpha}
\newcommand\ctindr{{\alpha_{\sss\rm r}}}
\newcommand\ctindpm{{\alpha_{\splusminus}}}
\newcommand\ctindp{{\alpha_{\splus}}}
\newcommand\ctindm{{\alpha_{\sminus}}}
\newcommand\matct{{\cal C}}
\newcommand\matcti{{\cal C}^{(\ctindex)}}
\newcommand\matctiP{\bar{\cal C}^{(\ctindex)}}

\newcommand\GenSh{{\cal F}}
\newcommand\GenShV{{\cal F}_{\sss\rm V}}
\newcommand\GenShVT{{\cal F}_{\sss\rm VT}}
\newcommand\GenShMC{{\cal F}_{\sss\rm MC}}
\newcommand\GenShMCatNLO{{\cal F}_{\sss\rm MC@NLO}}
\newcommand\GenShPOWHEG{{\cal F}_{\sss\rm POWHEG}}

\newcommand\PSnpo{\Phi_{n+1}}
\newcommand\PSn{\Phi_n}
\newcommand\Phit{\tilde{\Phi}}
%\newcommand\MapNLOi{{\bf M}^{(i,-)}}
%\newcommand\MapMCisr{{\bf S}^{\sss (ISR,+)}}
%\newcommand\MapMCfsr{{\bf S}^{\sss (FSR,+)}}
\newcommand\MapNLOi{{\bf M}^{(\ctindex)}}
\newcommand\MapNLOisro{{\bf M}^{\sss\rm (ISR_\splus)}}
\newcommand\MapNLOisrt{{\bf M}^{\sss\rm (ISR_\sminus )}}
\newcommand\MapNLOfsr{{\bf M}^{\sss\rm (FSR)}}
\newcommand\MapMCisr{{\bf S}^{\sss\rm (ISR)}}
\newcommand\MapMCfsr{{\bf S}^{\sss\rm (FSR)}}

%\newcommand\Kinnpo{{\bf K}_{n+1}}
%\newcommand\Kinn{{\bf K}_n}
%\newcommand\Kinnc{{\bf K}_n^c}
%\newcommand\Kinncp{{\bf K}_n^{c+}}
%\newcommand\Kinncm{{\bf K}_n^{c-}}
%\newcommand\Kinncpm{{\bf K}_n^{c\pm}}
%\newcommand\Kinm{{\bf K}_m}
\newcommand\Kinnpo{{\bf \Phi}_{n+1}}
\newcommand\Kinn{{\bf \Phi}_n}
\newcommand\BKinn{{\bf \bar{\Phi}}_n}
\newcommand\BKinm{{\bf \bar{\Phi}}_m}
\newcommand\Kinnc{{\bf \Phi}_n^c}
\newcommand\Kinncp{{\bf \Phi}_{n,\splus}}
\newcommand\Kinncm{{\bf \Phi}_{n,\sminus}}
\newcommand\Kinncpm{{\bf \Phi}_{n,\splusminus}}
\newcommand\BKinncp{\bar{\bf \Phi}_{n,\splus}}
\newcommand\BKinncm{\bar{\bf \Phi}_{n,\sminus}}
\newcommand\BKinncpm{\bar{\bf \Phi}_{n,\splusminus}}
\newcommand\Kinm{{\bf \Phi}_m}
\newcommand\Kinmpo{{\bf \Phi}_{m+1}}
\newcommand\Kinnt{{\bf \tilde{\Phi}}_n}
\newcommand\Kinnpot{{\bf \tilde{\Phi}}_{n+1}}

\newcommand\KinnpoMC{{\bf \Phi}_{n+1}^{({\sss\rm MC})}}
\newcommand\KinnMC{{\bf \Phi}_n^{({\sss\rm MC})}}
\newcommand\KinncpMC{{\bf \Phi}_{n,\splus}^{({\sss\rm MC})}}
\newcommand\KinncmMC{{\bf \Phi}_{n,\sminus}^{({\sss\rm MC})}}

\newcommand\KinnPH{{\bf \Phi}_n^{({\sss\rm PH})}}
\newcommand\KinncpPH{{\bf \Phi}_{n,\splus}^{({\sss\rm PH})}}
\newcommand\KinncmPH{{\bf \Phi}_{n,\sminus}^{({\sss\rm PH})}}
\newcommand\KinntPH{{\bf \tilde{\Phi}}_n^{({\sss\rm PH})}}

\newcommand\stepf{\theta}
\newcommand\Dfun{{\cal D}}
\newcommand\Sfun{{\cal S}}
\newcommand\Sin{{\cal S}^{(\INm)}}
\newcommand\Sout{{\cal S}^{(\OUTm)}}
%\newcommand\Sz{{\cal S}^{(0)}}
%\newcommand\So{{\cal S}^{(1)}}
%\newcommand\Szi{{\cal S}_i^{(0)}}
%\newcommand\Soij{{\cal S}_{ij}^{(1)}}
\newcommand\Sz{{\cal S}}
\newcommand\So{{\cal S}}
\newcommand\Szi{{\cal S}_i}
\newcommand\Szip{{\cal S}_i^\splus}
\newcommand\Szim{{\cal S}_i^\sminus}
\newcommand\Szipm{{\cal S}_i^\splusminus}
\newcommand\Szimp{{\cal S}_i^\sminusplus}
\newcommand\Soij{{\cal S}_{ij}}
\newcommand\Soji{{\cal S}_{ji}}
\newcommand\mydot{\!\cdot\!}
\newcommand\xiic{\left(\frac{1}{\xi_i}\right)_{\!\!\xi_c}}
\newcommand\xic{\left(\frac{1}{\xi}\right)_{\!\!\xi_c}}
\newcommand\lxic{\left(\frac{\log\xi}{\xi}\right)_{\!\!\xi_c}}
\newcommand\omzc{\left(\frac{1}{1-z}\right)_{\!\!\xi_c}}
\newcommand\lomzc{\left(\frac{\log(1-z)}{1-z}\right)_{\!\!\xi_c}}
\newcommand\omyid{\left(\frac{1}{1-y_i}\right)_{\!\!\deltaI}}
\newcommand\opyid{\left(\frac{1}{1+y_i}\right)_{\!\!\deltaI}}
\newcommand\ompyid{\left(\frac{1}{1\mp y_i}\right)_{\!\!\deltaI}}
\newcommand\omyjd{\left(\frac{1}{1-y_j}\right)_{\!\!\deltaO}}
\newcommand\omyijd{\left(\frac{1}{1-y_{ij}}\right)_{\!\!\deltaO}}
\newcommand\ompyd{\left(\frac{1}{1\mp y}\right)_{\!\!\delta}}
\newcommand\Prec{P_{\rm rec}}
\newcommand\Mrec{M_{\rm rec}}
\newcommand\Pboost{K_{\rm rec}}
%\newcommand\boosted[1]{{\underline #1}}
%\newcommand\mmod[1]{\boldsymbol{#1}}
\newcommand\mmod[1]{\underline{#1}}
\newcommand\MCatNLO{{\tt MC@NLO}}
\newcommand\pMCatNLO{{pMC@NLO}}
\newcommand\xicut{\xi_{c}}
\newcommand\deltaI{\delta_{\sss\rm I}}
\newcommand\deltaO{\delta_{\sss\rm O}}
\newcommand\MSB{{\rm \overline{MS}}}
\newcommand\DIG{{\rm DIS}_\gamma}
\newcommand\CA{C_{\sss\rm A}}
\newcommand\DA{D_{\sss\rm A}}
\newcommand\CF{C_{\sss\rm F}}
\newcommand\TF{T_{\sss\rm F}}
\newcommand\NC{N_{\rm c}}
\newcommand\NCt{N_{\rm c}^2}
\newcommand\NF{n_{\rm f}}

%\newcommand\APl{{<}}
\newcommand\APl{{reg}}
\newcommand\APreg{\hat{P}}
\newcommand\indsing{{i_s}}
\newcommand\fr{{f_r}}
\newcommand\fb{{f_b}}
\newcommand\Fb{{F_b}}
\newcommand\POWHEG{{POWHEG}}

\newcommand\Rad{\Phi_{\rm rad}}
\newcommand\Xrad{X_{\rm rad}}
%\newcommand\BRad{\bar{\Phi}_{\rm rad}}

\newcommand\muF{\mu_{\sss\rm F}}
\newcommand\muR{\mu_{\sss\rm R}}

\newcommand\frindsing{{\ctindr}}

\newcommand\qb{\bar{q}}
\newcommand\RED{}
\def\ord#1{{\cal O}\(#1\)}
\def\ga{\gamma}
\def\sirave{\overline{\sigma}_{\rm real} }
\def\siNLOave{\overline{\sigma}_{\rm NLO} }
\def\siMSBave{\overline{\sigma}_{\rm NLO}^{\rm \overline{MS}}}
\def\sicoll{\sigma^{\rm coll}_{\rm ct}}
\def\sibave{\overline{\sigma}_{\rm Born} }
\def\sivave{\overline{\sigma}_{\rm virt} }

\newcommand\xpm{{x_{\splus\sminus}}}
\newcommand\xipm{{x_{i,\splus\sminus}}}
\newcommand\vplus{{\tilde{v}_\splus}}
\newcommand\vminus{{\tilde{v}_\sminus}}
\newcommand\vpm{{\tilde{v}_\splusminus}}

\def\reff#1{(\ref{#1})}
\def\nn{\nonumber}

\title{Calculation of $\matSV_{\sss\rm fin}$ in the FKS framework:\\
an example using POWHEG-BOX scales}
\author{~}
%\date{~}
\begin{document}
\maketitle
\begin{abstract}{}
We show how to calculate the $\matSV_{\sss\rm fin}$ term needed by the
POWHEG-BOX program, starting from a generic expression for the virtual
term $\mathcal{V}_{b}$.  We show this in the case of DY processes.
\end{abstract}


We start from FNO, eq.7.190, which is taken from the original paper by
Altarelli, Ellis and Martinelli:
\begin{equation}
\label{eq:virtual}
  \mathcal{V}_{b,\,q\qb} = \left( \frac{4 \pi \muR^2}{s}
  \right)^{\epsilon} \frac{\Gamma (1 + \epsilon) \, \Gamma (1 -
  \epsilon)^2}{\Gamma (1 - 2 \epsilon)}  \frac{\as}{2 \pi}\, \CF \left[ -
  \frac{2}{\epsilon^2} - \frac{3}{\epsilon} - 8 + \pi^2
  \right]\mathcal{B}_{\,q\qb},
\end{equation}
Here $\muR$ stands for the renormalization scale, and
$s=2(k_\splus\cdot k_\sminus)$. 
% As in FNO, we start by observing
%that:
%\begin{equation}
%  \frac{\Gamma (1 + \epsilon) \,\Gamma (1 - \epsilon)^2}{\Gamma (1 - 2
%  \epsilon)} = \frac{1}{\Gamma (1 - \epsilon)} +\mathcal{O}\(\epsilon^3\),
%\end{equation}
%Therefore, up to terms that vanish in the limit $\epsilon\to 0$, we have:
%\begin{equation}
%  \mathcal{V}_{b,\,q\qb} = \left( \frac{4 \pi \mu^2}{s}
%  \right)^{\epsilon} \frac{\Gamma (1 + \epsilon) \, \Gamma (1 -
%  \epsilon)^2}{\Gamma (1 - 2 \epsilon)}  \frac{\as}{2 \pi}\, \CF \left[ -
%  \frac{2}{\epsilon^2} - \frac{3}{\epsilon} - 8 + \pi^2
%  \right]\mathcal{B}_{\,q\qb},
%\end{equation}
To extract $\matSV_{\sss\rm fin}$ from eq.~\reff{eq:virtual},
we need to equal the expression in~\reff{eq:virtual} to the following one
(FNO, eq.2.92):
\begin{equation} 
\matVb={\cal N}\,\asotpi
\Bigg[-\sum_{i \in \matI} \left(\frac{1}{\ep^2} \,C_{f_i}+
\frac{1}{\ep} \,\gamma_{f_i}\right)\matB
+\frac{1}{\ep}\sum_{\stackrel{i,j \in \matI}{i\ne j}}\log\frac{2k_i\mydot
  k_j}{Q^2}\, 
\matB_{ij}+\matSV_{\sss\rm fin}\Bigg]\,
\label{eq:FKSoneloop}
\end{equation}
where
\begin{equation}\label{eq:virtnorm}
{\cal N}=
\frac{(4\pi)^\ep}{\Gamma(1-\ep)}
\left(\frac{\muR^2}{Q^2}\right)^\ep.
\end{equation}

For the case at hand, equation~\reff{eq:FKSoneloop} reduces to
\begin{equation}
\mathcal{V}_{b,\,q\qb}=\(\frac{4\pi\muR^2}{Q^2}\)^\ep\frac{1}{\Gamma(1-\ep)}
\,\asotpi
\Bigg[-\(\frac{2}{\ep^2} +\frac{3}{\ep} -\frac{2}{\ep}\log\frac{s}{Q^2}\)
\CF \matB_{q\qb}  + \matSV_{{\sss\rm fin},\,q\qb}\Bigg]\,.
\label{eq:FKSoneloop_DY}
\end{equation}
where we used the fact that the only colored partons involved are the
two incoming quarks, and we also have that
$\matB^{q\qb}_{\splus\sminus}=\CF\matB_{q\qb}$.

We notice that $Q^2$ is here an \emph{arbitrary} scale.
In the FKS approach, the soft-virtual term is obtained summing to
$\matSV_{\sss\rm fin}$ a quantity that is $Q^2$ dependent
(see FNO, eq. 2.99, 2.100 and 2.101).
In the POWHEG-BOX code, this operation is authomatized
and it is performed in a file that the user is not supposed
to edit. 
Therefore, an explicit choice for this scale was needed. Since we choosed
\begin{equation}
Q^2=\muR^2\,,
\end{equation}
it is important to use the same value for $Q^2$ in eq.~\reff{eq:FKSoneloop_DY},
in order to extract a consistent expression for $\matSV_{\sss\rm fin}$.

Therefore we evaluate the expression in~\reff{eq:FKSoneloop_DY} with
$Q^2=\muR^2$, and equal it to the r.h.s. of
eq.~\reff{eq:virtual}, obtaining:
\begin{eqnarray}
&& (4\pi)^\ep\frac{1}{\Gamma(1-\ep)}
\Bigg[-\(\frac{2}{\ep^2} +\frac{3}{\ep} -\frac{2}{\ep}\log\frac{s}{\muR^2}\)
\matB_{q\qb}  + \frac{\matSV_{{\sss\rm fin},\,q\qb}}{\CF}\Bigg] =\phantom{\hspace{3cm}} \nn \\
&& \hspace{3cm}\left( \frac{4 \pi \muR^2}{s} \right)^{\epsilon} 
\frac{\Gamma (1 + \epsilon) \, \Gamma (1 -
  \epsilon)^2}{\Gamma (1 - 2 \epsilon)}  \left[ -
  \frac{2}{\epsilon^2} - \frac{3}{\epsilon} - 8 + \pi^2
  \right]\mathcal{B}_{\,q\qb}\,.
\end{eqnarray}
By using the fact that
\begin{equation}
  \frac{\Gamma (1 + \epsilon) \,\Gamma (1 - \epsilon)^2}{\Gamma (1 - 2
  \epsilon)} = \frac{1}{\Gamma (1 - \epsilon)} +\mathcal{O}\(\epsilon^3\),
\end{equation}
we have
\begin{equation}
\frac{\matSV_{{\sss\rm fin},\,q\qb}}{\CF}=
\left[ \left( \frac{\muR^2}{s} \right)^{\epsilon} \left( -\
  \frac{2}{\epsilon^2} - \frac{3}{\epsilon} - 8 + \pi^2
  \right) +
\left(\frac{2}{\ep^2} + \frac{3}{\ep} - \frac{2}{\ep}\log\frac{s}{\muR^2} \right)\right]
\matB_{q\qb}\,,
\end{equation}
that reduces to
\begin{equation}
\matSV_{{\sss\rm fin},\,q\qb}=\matB_{q\qb} \lq \pi^2 -8 -3\log\frac{\muR^2}{s}
-\log^2\frac{\muR^2}{s} \rq \CF\,
\label{eq:FKS_Vfinite_DY}
\end{equation}
after expanding $(\muR^2/s)^\ep$ as usual.
The expression in eq.~\reff{eq:FKS_Vfinite_DY} is the one needed by
the POWHEG-BOX program, and it has to be coded inside the subroutine
\tmtexttt{setvirtual}.

\end{document}



\documentclass{article}
\usepackage{amsmath}

%%%%%%%%%% Start TeXmacs macros
%\newcommand{\bignone}{}
%\newcommand{\tmmathbf}[1]{\ensuremath{\boldsymbol{#1}}}
%\newcommand{\tmop}[1]{\ensuremath{\operatorname{#1}}}
%%%%%%%%%% End TeXmacs macros

\begin{document}

\title{AAA}
%\title{}
%\author{}\maketitle

ddddd

\end{document}

We deal with an integral of the form
\begin{equation}
  I = \int_{- 1}^1 d y \int_0^{X (y)} d \xi \frac{1}{(1 - y)_+}  \left(
  \frac{1}{\xi} \right)_{\xi_c} f (\xi, y) .
\end{equation}
First of all, we demonstrate the identity
\begin{equation}
  \[ \[ \int_0^X d \xi \left( \frac{1}{\xi} \right)_{\xi_c} \bignone F (\xi) =
        \int_{0^{}}^1 d \tilde{\xi}  \left[ \left( \frac{1}{\tilde{\xi}}
        \right)_+ + \log \frac{X}{\xi_c} \bignone \delta ( \tilde{\xi})
        \right] F (\xi) \] \]
\end{equation}
with $\xi = X \tilde{\xi}$. If $F (0) = 0$ we simply have
\begin{equation}
  \int_0^X d \xi \left( \frac{1}{\xi} \right)_{\xi_c} \bignone F (\xi) =
  \int_{0^{}}^1 d \tilde{\xi}  \left( \frac{1}{\tilde{\xi}} \right)_+ F (\xi)
\end{equation}
For $F (\xi) = 1$ we have
\begin{equation}
  \int_0^X d \xi \bignone  \left( \frac{1}{\xi} \right)_{\xi_c} =
  \int_{\xi_c}^X \frac{d \xi}{\xi} = \bignone \log \frac{X}{\xi_c} =
  \int_{0^{}}^1 d \tilde{\xi}  \left[ \left( \frac{1}{\tilde{\xi}} \right)_+ +
  \log \frac{X}{\xi_c} \bignone \delta ( \tilde{\xi}) \right],
\end{equation}
which is obvious.

We now work out the integral. We follow only the $1 / (1 - y)$ term for
simplicity
\begin{eqnarray}
  I = \int_{- 1^{}}^1 d y \int_0^1 d \tilde{\xi}  \frac{1}{(1 - y)_+}  \left[
  \left( \frac{1}{\tilde{\xi}} \right)_+ + \log \frac{X (y)}{\xi_c} \bignone
  \delta ( \tilde{\xi}) \right] f (\xi, y) &  &  \nonumber\\
  = \int_{- 1}^1 d y \int_0^1 d \tilde{\xi}  \frac{1}{1 - y}  \left[ \frac{f (
  \tilde{\xi} X (y), y) - f (0, y)}{\tilde{\xi}} - \frac{f ( \tilde{\xi} X
  (1), 1) - f (0, 1)}{\tilde{\xi}} \left]  \right. \right. &  &  \nonumber\\
  + \int_{- 1}^1 d y \frac{1}{1 - y} \left[ \log \frac{X (y)}{\xi_c} f (0, y)
  - \log \frac{X (1)}{\xi_c} f (0, 1) \right], &  & 
\end{eqnarray}
which is our needed formula.

The real contribution to $\tilde{B}$ is obtained as
\begin{eqnarray}
  \bar{B}_{\tmop{real}} & = & \int d \tmmathbf{\Phi}_n \int_0^{2 \pi} d \phi
  \int_{- 1}^1 d y \int_0^{X (y)} d \xi J_{\tmop{rad}} (\xi, y, \phi) R
  \nonumber\\
  & = & \int d \tmmathbf{\Phi}_n \int_0^{2 \pi} d \phi \int_{- 1}^1 d y
  \int_0^{X (y)} d \xi \frac{1}{(1 - y)_+}  \left( \frac{1}{\xi}
  \right)_{\xi_c}  \frac{J_{\tmop{rad}} (\xi, y, \phi)}{\xi} (1 - y) \xi^2 R
  \nonumber\\
  & = & \int d \tmmathbf{\Phi}_n \int_0^{2 \pi} d \phi \int_{- 1}^1 d y
  \left[ \int_0^1 d \tilde{\xi}  \frac{1}{(1 - y)_+}  \left(
  \frac{1}{\tilde{\xi}} \right)_+  \frac{J_{\tmop{rad}} (\xi, y, \phi)}{\xi}
  (1 - y) \xi^2 R \right. \nonumber\\
  &  & \left. + \log \frac{X (y)}{\xi_c} \lim_{\xi \rightarrow 0} \left(
  \frac{J_{\tmop{rad}} (\xi, y, \phi)}{\xi} (1 - y) \xi^2 R \right) \right], 
\end{eqnarray}
where we should not forget that $\xi = \tilde{\xi} X (y)$ is also a function
of $y$. Defining
\begin{equation}
  f (\xi, y) = \frac{J_{\tmop{rad}} (\xi, y, \phi)}{\xi} (1 - y) \xi^2 R,
\end{equation}
we get
\begin{eqnarray}
  \bar{B}_{\tmop{real}} & = & \int d \tmmathbf{\Phi}_n \int_0^{2 \pi} d \phi
  \int_{- 1}^1 \frac{d y}{1 - y} \left\{ \int_0^1 d \tilde{\xi}  \left[
  \frac{f ( \tilde{\xi} X (y), y) - f (0, y)}{\tilde{\xi}} - \frac{f (
  \tilde{\xi} X (1), 1) - f (0, 1)}{\tilde{\xi}} \left]  \right. \right.
  \right. \nonumber\\
  &  & \left. + \left[ \log \frac{X (y)}{\xi_c} f (0, y) - \log \frac{X
  (1)}{\xi_c} f (0, 1) \right] \right\} . 
\end{eqnarray}

\end{document}
